% !TEX program = xelatex

\documentclass[12pt,a4paper]{article}
\usepackage[UTF8]{ctex}
\usepackage{float}
\usepackage{amsmath}
\usepackage{enumerate}
\usepackage{booktabs}
\usepackage{graphicx}
\usepackage{caption}
\usepackage{longtable}
\usepackage{pgfplots}
\usepackage[a4paper, left=3.18cm, right=3.18cm, top=2.54cm, bottom=2.54cm]{geometry}

\title{人工智能导论 \\ 情感分析实验报告}
\author{2017011620 计73 李家昊}
\date{\today}

\begin{document}

\captionsetup[figure]{labelfont={bf}, name={Figure}}
\captionsetup[table]{labelfont={bf}, name={Table}}
\maketitle

\section{数据处理}

\subsection{文本处理}

给定的数据集已经完成了分词,因此这里使用keras.preprocessing.text中的Tokenizer工具,统计文本中出现的所有词,按照词频从高到低,为每个词建立一个整数索引,然后将文本转换为整数序列。考虑到CNN模型要求各数据大小一致,这里将每篇文章的长度限制在600个词,对于过长的文章,截取其前600个词,对于过短的文章,则在其后补0,得到文本对应的整数序列,作为神经网络的输入。

\subsection{文本表示方法}

这里采用词向量(word
embedding)的方式表示文本。考虑到给定的数据集较小,不足以训练出较好的词向量,因此我参考了说明文档提供的预训练词向量下载地址,下载了基于搜狗新闻数据集训练的词向量(300维),将其用到了本次实验的embedding
layer中,并将其权值标记为non-trainable。

\subsection{标签表示方法}

这里采用分类问题的表示方法,将用户打分最高的类别作为整篇新闻的情感类别。

\subsection{训练集、验证集、测试集的说明}

在给定的训练集中,取16\%作为验证集(共375个样本),其余作为训练集(共1967个样本),给定的测试集(共2228个样本)仅作为最终测试使用,不参与任何训练过程,不发挥任何验证作用。

\section{模型结构}

\subsection{CNN}

这里参考了说明文档提供的Text CNN论文,构建Text
CNN模型,但模型结构与论文中的略有不同,如下图

\begin{figure}[H]
\centering
\includegraphics[width=\textwidth]{./model/text_cnn.png}
\caption{Text CNN Architecture}
\end{figure}

模型接受一个输入序列,先通过一个embedding layer,将输入序列转换为预训练好的词向量,然后分别用大小为 \(n=1,2,3,4,5\) 的卷积核对词向量做一维卷积,对应基于词的 \(n=1,2,3,4,5\) 元模型,每种大小的卷积核数量为128个,然后通过Max Pooling层,取出卷积的最大值,然后将每个通道内的五个pooling结果连接起来,形成feature map,将其flatten后,以0.5的概率进行dropout,然后用一层全连接层将其连接到输出,采用softmax函数计算出每个类别的预测概率。

\subsection{RNN}

\subsubsection{LSTM}

构建一个简单单向LSTM模型,结构如下图

\begin{figure}[H]
\centering
\includegraphics[width=0.48\textwidth]{./model/text_lstm.png}
\caption{LSTM Architecture}
\end{figure}

模型接受一个输入序列,先通过一个embedding layer,将输入序列转换为预训练好的词向量,然后通过一层有64个单元的LSTM层,再以0.5的概率进行dropout,最后用一层全连接层将其连接到输出,采用softmax函数计算出每个类别的预测概率。

\subsubsection{Bidirectional LSTM}

构建一个双向LSTM模型,结构如下图

\begin{figure}[H]
\centering
\includegraphics[width=0.7\textwidth]{./model/text_bi_lstm.png}
\caption{Bidirectional LSTM Architecture}
\end{figure}

模型接受一个输入序列,先通过一个embedding layer,将输入序列转换为预训练好的词向量,然后通过一层有128个单元的双向LSTM层(正向和逆向各64个单元),返回一个时间序列,接下来通过一个time distributed dense层,在时间维度上进行全连接,再通过maxpooling层得到每个通道的最大值,然后flatten降维,以0.5的概率进行dropout,最后用一层全连接层将其连接到输出,采用softmax函数计算出每个类别的预测概率。

\subsection{MLP (Baseline)}

构建多层感知机(MLP)作为baseline,网络结构如下

\begin{figure}[H]
\centering
\includegraphics[width=0.48\textwidth]{./model/text_mlp.png}
\caption{MLP Architecture}
\end{figure}

输入序列首先经过embedding层,转化为词向量,然后将其Flatten降维,经过一个全连接层,再以0.5的概率进行dropout,最后用一层全连接层将其连接到输出,采用softmax函数计算出每个类别的预测概率。

\section{实验结果}

训练模型时,采用Adam优化器,取learning rate为0.001,损失函数取为交叉熵。取batch size为256,训练100个epochs。训练过程中,总是保存验证集上loss最小的一个模型,用于最后的测试。

测试模型时,先加载上述方式保存的模型,然后在测试集上测试,最终测试结果如下表所示

\begin{longtable}[H]{llll}
\toprule
Model & Accuracy & F1-Score(Macro) & Coef.\tabularnewline
\midrule
\endhead
Text CNN & 62.97\% & 27.03\% & 61.61\%\tabularnewline
LSTM & 57.09\% & 16.62\% & 52.81\%\tabularnewline
Bidirectional LSTM & \textbf{63.42\%} & \textbf{31.07}\% & \textbf{62.17\%}\tabularnewline
MLP & 55.57\% & 17.86\% & 50.56\%\tabularnewline
\bottomrule
\caption{Results of implemented models}
\end{longtable}

特别说明:经统计,在测试集中有234条数据有多个最大标签,占总标签数的10.5\%。因此,在计算上表中的准确率时,只要模型预测出来的类别为最大标签之一时,即判定为预测正确。

此外,按照实验要求,F1-Score需要用Macro Average计算,但是由于数据集的缺陷,某些类别从未被模型预测过,其F1-Score被置为0,拉低了总体的F1-Score。因此,这里的F1-Score不具有参考意义。

\section{调整参数}

本次实验对Text CNN有所创新,受到GoogLeNet中Inception module的
\(1\times 1\) 卷积核的启发,我也将大小为1的一维卷积核用到了Text
CNN中,对应于词的一元模型。这么做的原因是,某些词的情感色彩非常明显,只要这些词出现在文章内,基本就可以确定这篇文章的情感类别。

下面进行对照实验,实现Baseline为包含长度为 \(n=2,3,4,5\) 的卷积核的Text CNN,与包含长度为 \(n=1,2,3,4,5\) 的卷积核的Text CNN做对比,在测试集上的测试结果如下

\begin{longtable}[]{llll}
\toprule
Model & Accuracy & F1-Score(Macro) & Coef.\tabularnewline
\midrule
\endhead
CNN(with 1-conv) & \textbf{62.97\%} & \textbf{27.03\%} & \textbf{61.61\%}\tabularnewline
CNN(baseline) & 62.06\% & 25.97\% & 60.21\%\tabularnewline
\bottomrule
\caption{Results of 1-conv CNN against baseline}
\end{longtable}

可见,增加了大小为1的一维卷积核后,准确率上升了约0.9\%,F1-Score与相关系数均有相应提升。

\section{问题思考}

\subsection{停止训练的时机}

我的做法:在训练集上以16\%的比例划分出验证集,训练过程中,总是保存验证集上loss最小的一个模型。然后使模型充分训练,当看到loss明显回升,且不会下降到更低点时,停止训练。然后取保存下来的模型进行测试。

固定迭代次数的方式:优点是方便实现,缺点是不太灵活,需要针对不同的模型选择不同的迭代次数,不能避免过拟合现象,无法预测测试集上的准确率。

通过验证集调整的方式:优点是容易观察到过拟合现象,以及模型的泛化能力,缺点是需要额外消耗计算资源,延长训练时间。

\subsection{参数初始化}

本实验中参数初始化方式是均匀分布初始化(uniform initialization)。初始化参数只要不是太大,一般来说对训练结果影响不大,最终都能收敛到合适的值。

零均值初始化能防止梯度爆炸问题,若初始值均值不为0,则可能产生梯度爆炸问题,导致模型无法训练。

高斯分布初始化能防止梯度消失问题,高斯分布初始化的权重集中在0点附近,多次经过sigmoid激活函数后,仍然能保持相应梯度,防止梯度消失。

正交初始化主要用在RNN的初始化,避免梯度爆炸和梯度消失的问题。

\subsection{防止过拟合的方式}

\begin{itemize}
    \item 增加 Dropout Layer。
    \item 增加 L1/L2 Regularization Layer。
    \item 增加 Batch Normalization Layer。
\end{itemize}

\subsection{CNN, RNN, MLP优缺点分析}

\subsubsection{CNN}

优点:训练速度快,参数数量少,不容易产生过拟合现象,考虑了上下文信息,大小为 $n$ 的一维卷积核对应了基于词的 $n$ 元模型。

缺点:需要预先确定矩阵的大小,训练期间不能发生变化,因此只能通过截长补短固定文本长度,但可能使长文本的重要信息丢失,无法达到训练效果。

\subsubsection{RNN}

优点:参数数量少,不容易产生过拟合现象,考虑了上下文信息,通过输入门、遗忘门、输出门完成对上下文信息的处理,不需要固定文本长度。

缺点:训练速度慢,实现复杂。

\subsubsection{MLP}

优点:实现简单,训练速度快。

缺点:直接将文本的词向量矩阵降维,会丢失上下文的信息,可能无法达到训练效果。此外,其参数数量较多,容易产生过拟合现象。

\section{对数据集的分析与建议}

本次实验的数据集在数量上和质量上都不尽人意,因此,我想对本实验的数据集提出如下建议:

\begin{itemize}

\item 扩大数据集的规模

    本次实验中,真正能用到训练中的样本仅为2000个左右,训练数据严重不足,非常容易出现过拟合现象。建议将训练集规模增大到10000个样本以上。

\item 提高数据集的标注质量

    我猜想这个数据集是直接从新浪网上用爬虫爬下来的,新浪网上选择情感分类的都是普通网友,并非专门的标注人士,网友的选择随机性很大,而且有些文章仅有几个网友投票,不能代表整篇文章的情感分类。

\item 使类别分配更加平均

    经过统计得出,训练集和测试集上各标签的数量如下图所示
  
    \pgfplotstableread[row sep=\\,col sep=&]
    {
        interval & train & test \\
        感动    & 416   & 333   \\
        同情    & 124   & 101   \\
        无聊    & 145   & 128   \\
        愤怒    & 984   & 1064  \\
        搞笑    & 367   & 348   \\
        难过    & 180   & 175   \\
        新奇    & 99    & 68    \\
        温馨    & 27    & 11    \\
    }\mydata

    \begin{figure}[H]
        \centering
        \begin{tikzpicture}
            \begin{axis}[
                ybar,
                bar width=.5cm,
                width=0.9\textwidth,
                height=0.6\textwidth,
                legend style={at={(0.8,1)}, anchor=north, legend columns=-1},
                symbolic x coords={感动,同情,无聊,愤怒,搞笑,难过,新奇,温馨},
                xtick=data,
                nodes near coords,
                nodes near coords align={vertical},
                ymin=0, ymax=1200,
                ylabel={Samples},
            ]
            \addplot table[x=interval, y=train]{\mydata};
            \addplot table[x=interval, y=test]{\mydata};
            \legend{Train Set, Test Set}
            \end{axis}
        \end{tikzpicture}
        \caption{Summary statistics of train set and test set}
    \end{figure}
    
    其中“愤怒”标签数占总标签数的 44.81\%,而“温馨”只占所有标签的0.83\%,标签分布极不均衡,导致某些类别从未被模型预测过,拉低了总体的F1-Score。

\end{itemize}

若能提升数据集的质量,将进一步提升训练效果,使训练结果更有说服力。

\newpage

\section{实验总结}

\begin{enumerate}
\def\labelenumi{\arabic{enumi}.}
\item
  通过本次实验,我实现了MLP, Text CNN,
  LSTM等文本处理模型,实现了文本的多分类任务,对神经网络的工作机制有了更深的理解,对F1-Score, Correlation Coefficient等评价指标更加熟悉。
\item
  通过调参实验培养了耐心和毅力,同时领悟了不少调参的经验,例如事先预估可训练参数的数量、通过dropout防止过拟合、通过调整learning
  rate实现精细调整等等,这些经验有效地提升了模型效果,其中双向LSTM效果最佳,准确率达到了63.42\%。
\item
  感谢助教的耐心指导!
\end{enumerate}

\end{document}
