% !TEX program = xelatex

\documentclass[12pt,a4paper]{article}
\usepackage[UTF8]{ctex}
\usepackage{float}
\usepackage{amsmath}
\usepackage{amsfonts}
\usepackage{enumerate}
\usepackage{booktabs}
\usepackage{graphicx}
\usepackage{longtable}
\usepackage{subfigure}

\usepackage{url}
\usepackage{multirow}

% for plotting 
\usepackage{caption}
\usepackage{pgfplots}

% for pseudo code 
\usepackage{algorithm}
\usepackage[noend]{algpseudocode}

% for reference 
\usepackage{hyperref}
\usepackage{cleveref}

% for code 
\usepackage{listings}
\usepackage{xcolor}
\usepackage{fontspec}
\definecolor{darkgreen}{rgb}{0,0.6,0}
\newfontfamily\consolas{Consolas}

\usepackage{amssymb}
\usepackage{pifont}
\usepackage{xcolor}
\newcommand{\cmark}{\ding{51}}
\newcommand{\xmark}{\ding{55}}

\lstset {
    basicstyle=\footnotesize\consolas, % basic font setting
    breaklines=true, 
    frame=single,     % {single, shadowbox, bottomline}
    keywordstyle=\color{blue}, 
    commentstyle=\color{darkgreen},
    stringstyle=\color{red},
    showstringspaces=false,
    % backgroundcolor=\color{black!5}, % set backgroundcolor
    numbers=left, 
    numberstyle=\ttfamily,
}

% Microsoft Word A4 paper default layout 
\usepackage[a4paper, left=3.18cm, right=3.18cm, top=2.54cm, bottom=2.54cm]{geometry}

% \captionsetup[figure]{labelfont={bf}, name={Figure}}
% \captionsetup[table]{labelfont={bf}, name={Table}}

\crefname{equation}{方程}{方程}
\Crefname{equation}{方程}{方程}
\crefname{table}{表}{表}
\Crefname{table}{表}{表}
\crefname{figure}{图}{图}
\Crefname{figure}{图}{图}

\title{数学实验:第十次作业}
\author{计算机系 \quad 计73 \quad 2017011620 \quad 李家昊}
\date{\today}

% 实验报告格式的基本要求

% 系别、班级、学号、姓名

% 1 实验目的
% 2 题目
%   2.1 计算题:题号,算法设计(包括计算公式),程序,计算结果(计算机输出),结果分析,结论。
%   2.2 应用题:题号,问题分析,模型假设,模型建立,算法设计(包括计算公式),程序,计算结果(计算机输出),结果的数学分析,结果的实际意义,结论。
% 3 收获与建议

% Calc
% \subsubsection{算法设计}
% \subsubsection{程序}
% \subsubsection{计算结果}
% \subsubsection{结果分析}
% \subsubsection{结论}

% App
% \subsubsection{问题分析}
% \subsubsection{模型假设}
% \subsubsection{模型建立}
% \subsubsection{算法设计}
% \subsubsection{程序}
% \subsubsection{计算结果}
% \subsubsection{结果的数学分析}
% \subsubsection{结果的实际意义}
% \subsubsection{结论}

\begin{document}

\maketitle

\section{实验目的}

\begin{itemize}
    \item 了解回归分析的基本原理,掌握MATLAB实现的方法。
    \item 练习用回归分析解决实际问题。
\end{itemize}

\section{问题求解}

\subsection{Chap13-Ex7 有氧锻炼(计算题)}

\subsubsection{问题分析}

题目设置了一个对照实验,分为胃溃疡病人和正常人两组,每组给出了30人的溶菌酶含量数据,需要确定病人和正常人的溶菌酶含量有无显著差别,这是一个两总体均值的假设检验问题。

\subsubsection{模型假设}

为了简化实际情况,模型基于以下假设,
\begin{enumerate}
    \item 正常人和胃溃疡病人的溶菌酶含量均服从正态分布。
    \item 除特别说明外,原始数据均准确无误。
\end{enumerate}

\subsubsection{模型建立}

由于样本量较小,未能通过总体分布的正态性检验,但是根据生活经验,很多医学现象都是服从正态分布的,因此这里假设病人和正常人的溶菌酶含量两个总体均服从正态分布,病人的溶菌酶含量服从$N(\mu_1, \sigma_1^2)$,正常人的溶菌酶含量服从$N(\mu_2, \sigma_2^2)$,使用两总体均值的假设检验方法,记原假设$H_0$和对立假设$H_1$为,
\begin{equation}
    H_0: \mu_1 = \mu_2, \quad H_1: \mu_1 \ne \mu_2
\end{equation}

由于总体方差未知,因此采用$t$检验。若原假设$H_0$被接受,则病人与正常人的溶菌酶含量无显著差别;若被拒绝,则有显著差别。

\subsubsection{算法设计}

在MATLAB的实现中,可采用\texttt{ttest2}命令进行两总体均值的假设检验。

\subsubsection{程序}

请参见附录\ref{sec:ex7_code}。

\subsubsection{计算结果}

使用病人组的全部数据时,用点估计求得总体分布为$N(15.33,16.45^2)$。经过$t$检验,得到$p$值为0.0251,低于默认的显著性水平0.05,原假设被拒绝,表明病人与正常人的溶菌酶含量有显著差别。

去掉病人组的最后5个数据后,用点估计求得总体分布为$N(11.51,11.75^2)$。经过$t$检验,得到$p$值为0.1558,高于默认的显著性水平0.05,原假设被接受,表明病人与正常人的溶菌酶含量无显著差别。

\subsubsection{结果的数学分析}

经过计算,病人组的最后5个数据很可能有误,因为其中包含了2个$3\sigma$之外,1个$2\sigma$之外的离群点,使参数估计产生了较大误差。为了减小误差,应当尽可能降低原始数据的噪声。

\subsubsection{结果的实际意义}

该计算结果具有一定的实际意义,可作为分析胃溃疡病理的重要参考。然而,题目给出的样本量相对较小,得出的结论不一定准确,在实际情况下,应当增大样本量,并保证原始数据准确无误,从而得出更可靠的结论。

\subsubsection{结论}

使用病人组的全部数据计算得出,病人与正常人的溶菌酶含量有显著差别;去掉病人组的最后5个数据后计算得出,病人与正常人的溶菌酶含量无显著差别。


\subsection{Chap13-Ex9 洗衣粉泡沫(计算题)}

% 9.(原油采购与加工)某公司用两种原油(A 和B)混合加工成两种汽油(甲和乙)。甲、乙两种汽油含原油A 的最低比例分别为50\%和60\%,每吨售价分别为4,800 元和5,600 元。该公司现有原油A 和B 的库存量分别为500 吨和1,000 吨,还可以从市场上买到不超过1,500 吨的原油A。原油A 的市场价为:购买量不超过500 吨时的单价为10,000 元/吨;购买量超过500吨但不超过1,000 吨时,超过500 吨的部分8,000 元/吨;购买量超过1,000 吨时,超过1,000 吨的部分6,000 元/吨。该公司应如何安排原油的采购和加工?请分别建立连续规划和整数规划模型来求解这个问题。

\subsubsection{问题分析}

该模型设计了一个生产场景,给出了两种原料的库存,原料采购的价格曲线,以及两种产品的成分约束以及市场价格,需要建立连续规划模型和整数规划模型。

\subsubsection{模型假设}

为了简化实际情况,模型基于以下假设,
\begin{enumerate}
    \item 混合过程中无物料损失,混合物能稳定共存。
    \item 原油采购是公司唯一的成本来源,汽油产品是唯一的收益来源。
    \item 汽油产品能在短时间内销售完毕。
\end{enumerate}

\subsubsection{模型建立}

\paragraph{连续规划} 设原油A中有$p$吨用于生产汽油甲,有$q$吨用于生产汽油乙,原油B中有$r$吨用于生产汽油甲,有$s$吨用于生产汽油乙,公司额外购买的原油A共$u$吨,购买总成本为$c$万元。则公司生产的汽油甲共$p+r$吨,乙共$q+s$吨。

汽油A,B需要满足的比例约束为,
\begin{equation}\label{eq:ex9_cons_prop}
    \frac{p}{p+r} \ge 50\%, \quad \frac{q}{q+s} \ge 60\%
\end{equation}

原油的总量约束为,
\begin{equation}\label{eq:ex9_cons_total}
    p+q \le 500 + u, \quad r+s \le 1000
\end{equation}

购买原油的总成本是关于购买量的分段函数,其图像如\Cref{fig:ex9_cost},数学表达式为,
\begin{equation}\label{eq:ex9_cons_seg}
    c = \begin{cases}
        u, & 0 \le u \le 500 \\
        0.8u + 100, & 500 < u \le 1000 \\
        0.6u + 300, & 1000 < u \le 1500
    \end{cases}
\end{equation}

还需要加上非负约束和取值范围约束,
\begin{equation}\label{eq:ex9_cons_range}
    p, q, r, s, u, c \ge 0, \quad u \le 1500
\end{equation}

需要最大化收益$f$,单位为万元,
\begin{equation}\label{eq:ex9_obj}
    \max f = 0.48(p+r) + 0.56(q+s) - c
\end{equation}

由于成本约束\Cref{eq:ex9_cons_seg}为非线性约束,因此这是一个非线性规划模型,决策变量为$p, q, r, s, u, c$,目标函数为\Cref{eq:ex9_obj},约束条件为\Cref{eq:ex9_cons_prop},\Cref{eq:ex9_cons_total},\Cref{eq:ex9_cons_seg}和\Cref{eq:ex9_cons_range}。

\paragraph{整数规划} 在连续规划的基础上,假设原油和汽油均为按吨买卖的,则需要增加整数约束,
\begin{equation}\label{eq:ex9_cons_int}
    p, q, r, s, u, c \in \mathbb{N}
\end{equation}

这就构成了一个整数规划模型,其决策变量为$p, q, r, s, u, c$,目标函数为\Cref{eq:ex9_obj},约束条件为\Cref{eq:ex9_cons_prop},\Cref{eq:ex9_cons_total},\Cref{eq:ex9_cons_seg},\Cref{eq:ex9_cons_range}和\Cref{eq:ex9_cons_int}。

\begin{figure}
    \centering
    \includegraphics[width=0.8\textwidth,trim={3.09cm 9.295cm 3.09cm 9.295cm},clip]{fig/ex9_cost.pdf}
    \caption{总购买费用(万元)随购买量(吨)的变化图像}
    \label{fig:ex9_cost}
\end{figure}

\subsubsection{算法设计}

对于连续规划和整数规划,可使用LINGO软件求解,对于非线性约束\Cref{eq:ex9_cons_seg},可采用\texttt{@if}语句处理分段函数。

\subsubsection{程序}

请参见附录\ref{sec:ex9_code}。

\subsubsection{计算结果}

\paragraph{连续规划} LINGO将该问题识别为NLP (Nonlinear Program),经过19次迭代,求得全局最优解,得到总收益$f$的最大值为500万元,各决策变量的最优值为,
\begin{equation}
    p = 0, \quad q = 1500, \quad r = 0, \quad s = 1000, \quad u = 1000, \quad c = 900
\end{equation}

\paragraph{整数规划} LINGO将该问题识别为PINLP (Pure Integer Nonlinear Program),经过315次迭代,求得全局最优解,得到总收益$f$的最大值为500万元,各决策变量的最优值为,
\begin{equation}
    p = 0, \quad q = 1500, \quad r = 0, \quad s = 1000, \quad u = 1000, \quad c = 900
\end{equation}

\subsubsection{结果的数学分析}

连续规划模型与整数规划模型求出了相同的全局最优解,但在迭代次数上,整数规划是连续规划的17倍。

这启示我们,对于整数规划模型,可以先去掉整数约束,化为连续规划模型,如果求得的最优解恰好是整数解,那么这同样也是整数规划模型的最优解,但求解的时间复杂度从非多项式降低到了多项式;如果不是整数解,才需要进一步求解整数规划问题。

\subsubsection{结果的实际意义}

该计算结果具有一定的实际意义,可作为制定生产方案的重要参考。在实际应用中,还需要考虑原料供应量和产品需求量,原料和产品的价格波动等现实因素,根据实际情况对模型进行调整,才能做出切实可用的模型。

例如最近国际原油出现了破天荒的负油价,对应到模型中,就需要去掉原油价格的非负约束,受新冠疫情影响,未来的汽油消费预期并不乐观,对应到模型中,就需要考虑汽油产品的市场需求量,多余的产能是不能带来收益的。

\subsubsection{结论}

该公司应当花费900万元采购1000吨原油A,将所有1500吨原油A和1000吨原油B全部用来生产汽油乙,此时收益最高,为500万元。


\subsection{Chap13-Ex13 Logistic模型(计算题)}

% 13.  Logistic 增长曲线模型和Gompertz增长曲线模型是计量经济学等学科中的两个常用模型,可以用来拟合销售量的增长趋势。
%     记Logistic 增长曲线模型为 y_t = L / (1 + a exp(-kt)),记Gompertz增长曲线模型为 y_t = L exp(-b exp(-kt)),这两个模型中L的经济学意义都是销售量的上限。表13.33中给出的是某地区高压锅的销售量(单位:万台),为给出此两模型的拟合结果,请考虑如下的问题:
% 1)	Logistic 增长曲线模型是一个可线性化模型吗。如果给定 L=3000,是否是一个可线性化模型,如果是,试用线性化模型给出参数a和k的估计值。
% 2)	利用1)所得到的a和k的估计值和L=3000作为Logistic模型的拟合初值,对Logistic模型做非线性回归。
% 3)	取初值L(0)=3000, b(0)=30, k(0)=0.4,拟合Gompertz模型。并与Logistic 模型的结果进行比较。

\subsubsection{算法设计}

\paragraph{第(1)问} Logistic增长曲线模型如下,
\begin{equation}
    y_t = \frac{L}{1 + a e^{-kt}}
\end{equation}

当参数$L,a,k$均未知时,该模型不是一个可线性化模型。当给定$L=3000$时,则是一个可线性化模型,可转化为,
\begin{equation}
    \ln\left(\frac{L}{y_t} - 1\right) = \ln a - kt
\end{equation}

其中$a,k$为待估参数。上式左端为因变量,记为$y$,自变量为$t$,令$\beta_0 = \ln a$且$\beta_1 = -k$,则上式可表示为$y = \beta_0 + \beta_1 t$。在MATLAB中,可采用\texttt{regress}命令进行线性回归。

\paragraph{第(2)问} 利用第(1)问所得到的$a$和$k$的估计值和$L=3000$作为Logistic模型的拟合初值,对Logistic模型做非线性回归。在MATLAB中,可采用\texttt{nlinfit}命令进行非线性回归。

\paragraph{第(3)问} Gompertz模型定义如下,
\begin{equation}
    y_t = L e^{-b e^{-kt}}
\end{equation}

取初值$L_0=3000, b_0=30, k_0=0.4$,用\texttt{nlinfit}命令进行非线性回归分析。

\subsubsection{程序}

请参见附录\ref{sec:ex13_code}。

\subsubsection{计算结果}

\paragraph{第(1)问} 经过计算,得到回归模型参数如\Cref{tab:ex13_logistic_linear}。

\begin{table}[H]
    \centering
    \caption{Logistic回归模型}
    \label{tab:ex13_logistic_linear}
    \begin{tabular}{|c|c|c|}
        \hline
        回归系数 & 估计值 & 置信区间\\
        \hline
        \hline
        \(\beta_0\) & 3.8032 & [3.5765, 4.0299]\\
        \hline
        \(\beta_1\) & -0.4941 & [-0.5262, -0.4621]\\
        \hline
        \multicolumn{3}{|c|}{$R^2=0.9905, \quad F=1150.7545, \quad p=0.0000, \quad s^2=0.0386$}\\
        \hline
    \end{tabular}
\end{table}

计算参数$a,k$的估计值,
\begin{equation}
    \hat{a} = e^{\hat{\beta}_0} = 44.8463, \quad \hat{k} = -\hat{\beta}_1 = 0.4941
\end{equation}

画出拟合曲线,如\Cref{fig:ex13_logistic_linear},经过计算,其剩余标准差为$s=5053.0508$。

\begin{figure}[H]
    \centering
    \includegraphics[width=0.8\textwidth,trim={3.09cm 9.295cm 3.09cm 9.295cm},clip]{fig/ex13_logistic_linear.pdf}
    \caption{Logistic线性回归模型拟合曲线}
    \label{fig:ex13_logistic_linear}
\end{figure}

\paragraph{第(2)问} 由第(1)问计算结果,可得非线性拟合的初值为$L_0 = 3000, a_0 = 44.8463, k_0 = 0.4941$,经过非线性回归,得到回归系数为,
\begin{equation}
    \hat{L} = 3260.4185, \quad \hat{a} = 30.5351, \quad \hat{k} = 0.4148
\end{equation}

画出拟合曲线,如\Cref{fig:ex13_logistic_nonlinear},其剩余标准差为$s=1765.1289$,模型的拟合效果比线性回归模型更好。

\begin{figure}[H]
    \centering
    \includegraphics[width=0.8\textwidth,trim={3.09cm 9.295cm 3.09cm 9.295cm},clip]{fig/ex13_logistic_nonlinear.pdf}
    \caption{Logistic非线性回归模型拟合曲线}
    \label{fig:ex13_logistic_nonlinear}
\end{figure}

\paragraph{第(3)问} 以$L_0=3000, b_0=30, k_0=0.4$作为初值,经过非线性回归,得到Gompertz模型参数的估计值为,
\begin{equation}
    \hat{L} = 4810.1269, \quad \hat{b} = 4.5920, \quad \hat{k} = 0.1747
\end{equation}

画出拟合曲线,如\Cref{fig:ex13_gompertz_nonlinear},其剩余标准差为$s=308.1378$,低于Logistic模型,说明Gompertz模型比Logistic模型的拟合效果更好。

\begin{figure}[H]
    \centering
    \includegraphics[width=0.8\textwidth,trim={3.09cm 9.295cm 3.09cm 9.295cm},clip]{fig/ex13_gompertz_nonlinear.pdf}
    \caption{Gompertz非线性回归模型拟合曲线}
    \label{fig:ex13_gompertz_nonlinear}
\end{figure}

\subsubsection{结果分析}

求解非线性回归模型时,选取合适的初值,可以降低计算量,加快收敛速度。如果模型可线性化,可以先求出线性化回归模型的参数估计值,将其作为初值,进一步求解非线性回归模型。

\subsubsection{结论}

Logistic增长曲线模型不是一个可线性化模型,如果给定$L=3000$,则是一个可线性化模型,用线性化模型给出参数$a$和$k$的估计值为$\hat{a} = 44.8463$和$\hat{k} = 0.4941$。

将上述$a$和$k$的估计值和$L=3000$作为Logistic模型的拟合初值,对Logistic模型做非线性回归,得到$\hat{L} = 3260.4185, \hat{a} = 30.5351, \hat{k} = 0.4148$。

取初值$L_0=3000, b_0=30, k_0=0.4$,拟合Gompertz模型,得到$\hat{L} = 4810.1269, \hat{b} = 4.5920, \hat{k} = 0.1747$,与Logistic模型相比,Gompertz模型的拟合效果更好。


\section{收获与建议}

在本次实验中,我掌握了回归分析的基本原理,用回归分析方法建立了实际问题的模型,并用MATLAB进行求解,在解决实际问题的过程中,我对数学方法的原理和应用有了更深刻的理解。

感谢老师和助教在这个学期的付出,我在这门课上收获颇丰,学到了很多应用数学的理论知识,掌握了用数学建模解决实际问题的基本方法,也锻炼了编程实践的能力。在以后的课程中,希望助教能对每次的实验进行详细的解答,希望老师在课堂上介绍更多数学应用的前沿知识。

\section{附录:程序代码}

\subsection{Chap13-Ex7}\label{sec:ex7_code}

\lstinputlisting[language=Matlab]{../src/ex7.m}

\subsection{Chap13-Ex9}\label{sec:ex9_code}

\lstinputlisting[language=Matlab]{../src/ex9.m}

\subsection{Chap13-Ex13}\label{sec:ex13_code}

\lstinputlisting[language=Matlab]{../src/ex13.m}

\end{document}