% !TEX program = xelatex

\documentclass[12pt,a4paper]{article}
\usepackage[UTF8]{ctex}
\usepackage{float}
\usepackage{amsmath}
\usepackage{amsfonts}
\usepackage{enumerate}
\usepackage{booktabs}
\usepackage{graphicx}
\usepackage{longtable}
\usepackage{subfigure}

\usepackage{url}
\usepackage{multirow}

% for plotting 
\usepackage{caption}
\usepackage{pgfplots}

% for pseudo code 
\usepackage{algorithm}
\usepackage[noend]{algpseudocode}

% for reference 
\usepackage{hyperref}
\usepackage{cleveref}

% for code 
\usepackage{listings}
\usepackage{xcolor}
\usepackage{fontspec}
\definecolor{darkgreen}{rgb}{0,0.6,0}
\newfontfamily\consolas{Consolas}

\lstset {
    basicstyle=\footnotesize\consolas, % basic font setting
    breaklines=true, 
    frame=single,     % {single, shadowbox, bottomline}
    keywordstyle=\color{blue}, 
    commentstyle=\color{darkgreen},
    stringstyle=\color{red},
    showstringspaces=false,
    % backgroundcolor=\color{black!5}, % set backgroundcolor
    numbers=left, 
    numberstyle=\ttfamily,
}

% Microsoft Word A4 paper default layout 
\usepackage[a4paper, left=3.18cm, right=3.18cm, top=2.54cm, bottom=2.54cm]{geometry}

% \captionsetup[figure]{labelfont={bf}, name={Figure}}
% \captionsetup[table]{labelfont={bf}, name={Table}}

\crefname{equation}{方程}{方程}
\Crefname{equation}{方程}{方程}
\crefname{table}{表}{表}
\Crefname{table}{表}{表}
\crefname{figure}{图}{图}
\Crefname{figure}{图}{图}

\title{数学实验:第三次作业}
\author{计算机系 \quad 计73 \quad 2017011620 \quad 李家昊}
\date{\today}

% 实验报告格式的基本要求

% 系别、班级、学号、姓名

% 1 实验目的
% 2 题目
%   2.1 计算题:题号,算法设计(包括计算公式),程序,计算结果(计算机输出),结果分析,结论。
%   2.2 应用题:题号,问题分析,模型假设,模型建立,算法设计(包括计算公式),程序,计算结果(计算机输出),结果的数学分析,结果的实际意义,结论。
% 3 收获与建议

% Calc
% \subsubsection{算法设计}
% \subsubsection{Matlab程序}
% \subsubsection{计算结果}
% \subsubsection{结果分析}
% \subsubsection{结论}

\begin{document}

\maketitle

\section{实验目的}

\begin{itemize}
    \item 学会用MATLAB软件数值求解线性代数方程组,对迭代法的收敛性和解的稳定性作初步分析;
    \item 通过实例学习用线性代数方程组解决简化的实际问题。
\end{itemize}

\section{问题求解}

\subsection{Chap5-Ex1 误差(计算题)}

\subsubsection{算法设计}

由题意,需要解下列方程,
\begin{equation}\label{eq:ex1_model}
    \boldsymbol{A}_1 \boldsymbol{x}_1 = \boldsymbol{b}_1, \quad \boldsymbol{A}_2 \boldsymbol{x}_2 = \boldsymbol{b}_2
\end{equation}
其中$\boldsymbol{A}$为Vandermonde矩阵,
\begin{equation}
    \boldsymbol{A}_1 = \left(
    \begin{matrix}
        1      & x_0    & x_0^2     & \cdots & x_0^{n-1} \\
        1      & x_1    & x_1^2     & \cdots & x_1^{n-1} \\
        \vdots & \vdots & \vdots    & \ddots & \vdots     \\
        1      & x_{n-1}& x_{n-1}^2 & \cdots & x_{n-1}^{n-1}
    \end{matrix}    
    \right),
    \quad 
    x_k = 1+0.1k, \quad k=0,1,\cdots,n-1
\end{equation}

$\boldsymbol{A}_2$为Hilbert矩阵,
\begin{equation}
    \boldsymbol{A}_2 = \left(
    \begin{matrix}
        1 & \frac{1}{2} & \frac{1}{3} & \cdots & \frac{1}{n} \\
        \frac{1}{2} & \frac{1}{3} & \ddots & \ddots & \vdots \\
        \frac{1}{3} & \ddots & \ddots & \ddots & \frac{1}{2n-3} \\
        \vdots & \ddots & \ddots & \frac{1}{2n-3} & \frac{1}{2n-2}\\
        \frac{1}{n} & \cdots & \frac{1}{2n-3} & \frac{1}{2n-2} & \frac{1}{2n-1}
    \end{matrix}
    \right)
\end{equation}

并且$\boldsymbol{b}_1$为$\boldsymbol{A}_1$的行和,$\boldsymbol{b}_2$为$\boldsymbol{A}_2$的行和。

显然,\Cref{eq:ex1_model}的解为,
\begin{equation}
    \boldsymbol{x}_1 = \boldsymbol{x}_2 = (1,1,\cdots,1)^T
\end{equation}

对于$\boldsymbol{x}$的数值解,用Matlab的左除命令求解即可。对于误差分析,可根据如下结论求出相对误差限。

当$\boldsymbol{A}$受扰动而$\boldsymbol{b}$不变时,若$||\boldsymbol{A}^{-1}||\cdot||\delta\boldsymbol{A}|| < 1$,则$\boldsymbol{x}$的相对误差限为,
\begin{equation}
    \frac{||\delta \boldsymbol{x}||}{||\boldsymbol{x}||} \le \frac{Cond(\boldsymbol{A})}{1-Cond(\boldsymbol{A})\cdot\dfrac{||\delta \boldsymbol{A}||}{||\boldsymbol{A}||}} \cdot \frac{||\delta \boldsymbol{A}||}{||\boldsymbol{A}||}
\end{equation}

当$\boldsymbol{b}$受扰动而$\boldsymbol{A}$不变时,$\boldsymbol{x}$的相对误差限为,
\begin{equation}
    \frac{||\delta \boldsymbol{x}||}{||\boldsymbol{x}||} \le Cond(\boldsymbol{A}) \cdot \frac{||\delta \boldsymbol{b}||}{||\boldsymbol{b}||}
\end{equation}

\subsubsection{Matlab程序}

请参见附录\ref{sec:ex1_code}。

\subsubsection{计算结果}

\paragraph{方程求解} 当$n=5$时,求解\Cref{eq:ex1_model}得,
\begin{equation}
    \boldsymbol{x}_1 = \boldsymbol{x}_2 = (1.0000,1.0000,1.0000,1.0000,1.0000)^T
\end{equation}
与预期结果相符。

\paragraph{条件数计算} 当$n=5,7,9$时,计算$\boldsymbol{A}_1$和$\boldsymbol{A}_2$的条件数如\Cref{tab:ex1_cond}所示,可以看出,随着$n$的增大,$\boldsymbol{A}_1,\boldsymbol{A}_2$的条件数随之增大,病态严重。此外,在相同的$n$下,$\boldsymbol{A}_2$的条件数比$\boldsymbol{A}_1$更大,因此$\boldsymbol{A}_2$的病态更严重。

\begin{table}
    \centering
    \caption{不同$n$值下矩阵$\boldsymbol{A}_1,\boldsymbol{A}_2$的条件数}
    \label{tab:ex1_cond}
    \begin{tabular}{c|ccc}
        \toprule
        $n$ & 5 & 7 & 9\tabularnewline
        \midrule
        $Cond(\boldsymbol{A}_1)$ & $3.5740\times 10^{5}$ & $8.7385\times10^{7}$ & $2.2739\times{10}^{10}$ \tabularnewline
        $Cond(\boldsymbol{A}_2)$ & $4.7661\times 10^{5}$ & $4.7537\times10^{8}$ & $4.9315\times{10}^{11}$ \tabularnewline
        \bottomrule
    \end{tabular}
\end{table}

\paragraph{扰动$\boldsymbol{A}_1,\boldsymbol{A}_2$后的计算结果及误差} 令$n=5,7,9$,$\boldsymbol{b}_1$和$\boldsymbol{b}_2$不变,在$\boldsymbol{A}_1(n,n)$和$\boldsymbol{A}_2(n,n)$上分别加上扰动$\varepsilon=10^{-10},10^{-8},10^{-6}$,限于篇幅,详细的计算结果请参见附录\ref{sec:ex1_details}中的\Cref{tab:ex1_a1}及\Cref{tab:ex1_a2}。经扰动得到的解记作$\tilde{\boldsymbol{x}}$,则相对误差为$\dfrac{||\tilde{\boldsymbol{x}} - \boldsymbol{x}||}{||\boldsymbol{x}||}$,$\boldsymbol{x}_1,\boldsymbol{x}_2$的相对误差计算结果分别如\Cref{tab:ex1_err1},\Cref{tab:ex1_err2}所示,其中相对误差限的计算省略了$||\boldsymbol{A}^{-1}||\cdot||\delta\boldsymbol{A}|| \ge 1$的情形。

\begin{table}
    \centering
    \caption{扰动$\boldsymbol{A}_1(n,n)$后$\boldsymbol{x}_1$的实际相对误差及相对误差限}
    \label{tab:ex1_err1}
    \begin{tabular}{c|cc|cc|cc}
        \toprule
        \multirow{2}{*}{\(\varepsilon \backslash n\)} & \multicolumn{2}{c}{5} & \multicolumn{2}{|c}{7} & \multicolumn{2}{|c}{9}\tabularnewline
        \cline{2-7}
        & Actual & Limit & Actual & Limit & Actual & Limit\tabularnewline
        \midrule
        \(10^{-10}\) & 0.0000 & 0.0000 & 0.0000 & 0.0003 & 0.0000 &
        0.0134\tabularnewline
        \(10^{-8}\) & 0.0000 & 0.0004 & 0.0003 & 0.0300 & 0.0033 &
        /\tabularnewline
        \(10^{-6}\) & 0.0021 & 0.0444 & 0.0319 & / & 0.3295 & /\tabularnewline
        \bottomrule
    \end{tabular}
\end{table}

\begin{table}
    \centering
    \caption{扰动$\boldsymbol{A}_2(n,n)$后$\boldsymbol{x}_2$的实际相对误差及相对误差限}
    \label{tab:ex1_err2}
    \begin{tabular}{c|cc|cc|cc}
        \toprule
        \multirow{2}{*}{\(\varepsilon \backslash n\)} & \multicolumn{2}{c}{5} & \multicolumn{2}{|c}{7} & \multicolumn{2}{|c}{9}\tabularnewline
        \cline{2-7}
        & Actual & Limit & Actual & Limit & Actual & Limit\tabularnewline
        \midrule
        \(10^{-10}\) & 0.0000 & 0.0000 & 0.0021 & 0.0295 & 0.7280 &
        /\tabularnewline
        \(10^{-8}\) & 0.0005 & 0.0031 & 0.1893 & / & 3.1999 & /\tabularnewline
        \(10^{-6}\) & 0.0490 & 0.4371 & 1.7383 & / & 3.3124 & /\tabularnewline
        \bottomrule
    \end{tabular}
\end{table}

\paragraph{扰动$\boldsymbol{b}_1,\boldsymbol{b}_2$后的计算结果及误差} 令$n=5,7,9$,$\boldsymbol{A}_1$和$\boldsymbol{A}_2$不变,在$\boldsymbol{b}_1(n)$和$\boldsymbol{b}_2(n)$上分别加上扰动$\varepsilon=10^{-10},10^{-8},10^{-6}$,限于篇幅,详细的计算结果请参见附录\ref{sec:ex1_details}中的\Cref{tab:ex1_b1}及\Cref{tab:ex1_b2}。$\boldsymbol{x}_1,\boldsymbol{x}_2$的相对误差计算结果分别如\Cref{tab:ex1_b1_err},\Cref{tab:ex1_b2_err}所示。

\begin{table}
    \centering
    \caption{扰动$\boldsymbol{b}_1(n)$后$\boldsymbol{x}_1$的实际相对误差及相对误差限}
    \label{tab:ex1_b1_err}
    \begin{tabular}{c|cc|cc|cc}
        \toprule
        \multirow{2}{*}{\(\varepsilon \backslash n\)} & \multicolumn{2}{c}{5} & \multicolumn{2}{|c}{7} & \multicolumn{2}{|c}{9}\tabularnewline
        \cline{2-7}
        & Actual & Limit & Actual & Limit & Actual & Limit\tabularnewline
        \midrule
        \(10^{-10}\) & 0.0000 & 0.0000 & 0.0000 & 0.0001 & 0.0000 &
        0.0068\tabularnewline
        \(10^{-8}\) & 0.0000 & 0.0002 & 0.0003 & 0.0137 & 0.0033 &
        0.6808\tabularnewline
        \(10^{-6}\) & 0.0021 & 0.0200 & 0.0319 & 1.3690 & 0.3303 &
        68.0786\tabularnewline
        \bottomrule
    \end{tabular}
\end{table}

\begin{table}
    \centering
    \caption{扰动$\boldsymbol{b}_2(n)$后$\boldsymbol{x}_2$的实际相对误差及相对误差限}
    \label{tab:ex1_b2_err}
    \begin{tabular}{c|cc|cc|cc}
        \toprule
        \multirow{2}{*}{\(\varepsilon \backslash n\)} & \multicolumn{2}{c}{5} & \multicolumn{2}{|c}{7} & \multicolumn{2}{|c}{9}\tabularnewline
        \cline{2-7}
        & Actual & Limit & Actual & Limit & Actual & Limit\tabularnewline
        \midrule
        \(10^{-10}\) & 0.0000 & 0.0000 & 0.0021 & 0.0124 & 0.9330 &
        11.1158\tabularnewline
        \(10^{-8}\) & 0.0005 & 0.0015 & 0.2103 & 1.2389 & 93.3044 &
        1111.5847\tabularnewline
        \(10^{-6}\) & 0.0512 & 0.1519 & 21.0324 & 123.8889 & 9330.4374 &
        111158.4664\tabularnewline
        \bottomrule
    \end{tabular}
\end{table}

\subsubsection{结果分析}

根据\Cref{tab:ex1_err1},\Cref{tab:ex1_err2},\Cref{tab:ex1_b1_err},\Cref{tab:ex1_b2_err}中的实际相对误差及相对误差限的计算结果,可以验证相对误差不超过其相对误差限。总体来看,实际相对误差的值主要集中在相对误差限的1\%到10\%的区间内,因此,利用相对误差限来估算大致的实际相对误差是可行的。

Vandermonde矩阵和Hilbert矩阵都是病态程度非常严重的矩阵,若矩阵或右端项稍有扰动,方程的解就会千差万别,例如,当规模$n=9$,扰动大小$\varepsilon = 10^{-6}$时,解的相对误差竟达到将近$10^4$级别,这样的结果在实际应用中是没有任何意义的。此外,当扰动$\varepsilon$越大,方程的解的相对误差限就越大,其实际相对误差在一般情况下也越大。当$n$越大,则两个矩阵的条件数越大,病态越严重,方程越不稳定,对扰动越敏感,从而越难获得精确的解。在相同的$n$值下,Hilbert矩阵的条件数比Vandermonde矩阵的条件数更大,病态更严重,在相同的扰动下,解的相对误差更大。

\subsubsection{结论}

矩阵的条件数越大,病态越严重,方程越不稳定,对扰动越敏感,从而越难获得精确的解;Vandermonde矩阵和Hilbert矩阵都是病态矩阵,在相同规模下,Hilbert矩阵的病态程度比Vandermonde矩阵更严重;在实际应用中,例如在神经网络的矩阵运算中,应当尽可能避免病态矩阵所带来的不稳定性,从而增强系统的鲁棒性。

\subsection{Chap5-Ex3 迭代法(计算题)}

\subsubsection{算法设计}

题目需要用Jacobi迭代法和Gauss-Seidel迭代法求解方程$\boldsymbol{Ax} = \boldsymbol{b}$,其中$\boldsymbol{A}\in \mathbb{R}^{n\times n}$,这里$n=20$,定义为,
\begin{equation}
    \boldsymbol{A} = \left(
        \begin{matrix}
            3 & -1/2 & -1/4 & & \\
            -1/2 & 3 & \ddots & \ddots & \\
            -1/4 & \ddots & \ddots & \ddots & -1/4 \\
            & \ddots & \ddots & 3 & -1/2 \\
            & & -1/4 & -1/2 & 3
        \end{matrix}
    \right)
\end{equation}

首先将$\boldsymbol{A}$分解为$\boldsymbol{D} - \boldsymbol{L} - \boldsymbol{U}$的形式,其中$\boldsymbol{D} = \mathrm{diag}(a_{11}, a_{22}, \cdots, a_{nn})$,
\begin{equation}
    \boldsymbol{L} = -\left(
    \begin{matrix}
        0 &  &  &  \\
        a_{21} & 0 & & \\
        \vdots & \ddots & \ddots & \\
        a_{n1} & \cdots & a_{n,n-1} & 0
    \end{matrix}    
    \right), \quad
    \boldsymbol{U} = -\left(
    \begin{matrix}
        0 & a_{12} & \cdots & a_{1n} \\
        & 0 & \ddots & \vdots \\
        & & \ddots & a_{n-1,n} \\
        & & & 0
    \end{matrix}
    \right)
\end{equation}

采用迭代法求解时,记迭代总数为$m$,在第$k$次迭代时,有状态方程,
\begin{equation}
    \boldsymbol{x}^{(k+1)} = \boldsymbol{B} \boldsymbol{x}^{(k)} + \boldsymbol{f}, \quad k = 0,1,2,\cdots,m-1
\end{equation}

采用Jacobi迭代法时,
\begin{equation}
    \boldsymbol{B} = \boldsymbol{D}^{-1}(\boldsymbol{L} + \boldsymbol{U}), \quad \boldsymbol{f} = \boldsymbol{D}^{-1}\boldsymbol{b}
\end{equation}

采用Gauss-Seidel迭代法时,
\begin{equation}
    \boldsymbol{B} = (\boldsymbol{D} - \boldsymbol{L})^{-1} \boldsymbol{U}, \quad \boldsymbol{f} = (\boldsymbol{D} - \boldsymbol{L})^{-1} \boldsymbol{b}
\end{equation}

如此,就可以利用迭代法求解线性方程组$\boldsymbol{Ax} = \boldsymbol{b}$了。对于迭代矩阵$\boldsymbol{B}$的谱半径,可通过\texttt{max(abs(eig(B)))}命令计算。

\subsubsection{Matlab程序}

请参见附录\ref{sec:ex3_code}。

\subsubsection{计算结果}

这里设置迭代误差限为$\varepsilon=10^{-5}$,当迭代误差$||\boldsymbol{x}^{(k+1)} - \boldsymbol{x}^{(k)}||_{\infty} < \varepsilon$时,算法结束,输出此时的迭代次数。

\paragraph{不同的初始向量和右端项} 选取不同的初始向量$\boldsymbol{x}^{(0)}$和右端项$\boldsymbol{b}$如\Cref{tab:ex3_x_b}所示。其中包括了常量向量,顺序递增向量,以及周期向量。在不同的初始向量和右端项下,采用Jacobi迭代法和Gauss-Seidel迭代法求解的迭代次数如\Cref{tab:ex3_iter}所示。

\begin{table}
    \centering
    \caption{不同的初始向量$\boldsymbol{x}^{(0)}$和右端项$\boldsymbol{b}$的选取}
    \label{tab:ex3_x_b}
    \begin{tabular}{c|ccc}
        \toprule
        \(i\) & 1 & 2 & 3\tabularnewline
        \midrule
        \(\boldsymbol{x}^{(0)}_i\) & \((1,1,...,1)^T\) & \((1,2,...,n)^T\) &
        \((-1,1,...,(-1)^n)^T\)\tabularnewline
        \(\boldsymbol{b}_i\) & \((1,1,...,1)^T\) & \((1,2,...,n)^T\) &
        \((-1,1,...,(-1)^n)^T\)\tabularnewline
        \bottomrule
    \end{tabular}
\end{table}

\begin{table}
    \centering
    \caption{在不同的初始向量$\boldsymbol{x}^{(0)}$和右端项$\boldsymbol{b}$下,给定迭代误差为$10^{-5}$时,Jacobi (J)和Gauss-Seidel (G-S)迭代法的迭代次数。}
    \label{tab:ex3_iter}
    \begin{tabular}{c|cc|cc|cc}
        \toprule
        \multirow{2}{*}{\(\boldsymbol{b} \backslash \boldsymbol{x}^{(0)}\)} &
        \multicolumn{2}{c}{\(\boldsymbol{x}^{(0)}_1\)} & \multicolumn{2}{|c}{\(\boldsymbol{x}^{(0)}_2\)} &
        \multicolumn{2}{|c}{\(\boldsymbol{x}^{(0)}_3\)}\tabularnewline
        \cline{2-7}
        & J & G-S & J & G-S & J & G-S\tabularnewline
        \midrule
        \(\boldsymbol{b}_1\) & 15 & 11 & 20 & 14 & 16 & 11\tabularnewline
        \(\boldsymbol{b}_2\) & 20 & 14 & 19 & 13 & 20 & 14\tabularnewline
        \(\boldsymbol{b}_3\) & 17 & 12 & 20 & 14 & 10 & 9\tabularnewline
        \bottomrule
    \end{tabular}
\end{table}

\paragraph{成倍增长$\boldsymbol{A}$的主对角线} 取定初始向量和右端项为,
\begin{equation}
    \boldsymbol{x}^{(0)} = (1,1,...,1)^T, \quad \boldsymbol{b} = (1,1,...,1)^T
\end{equation}
将$\boldsymbol{A}$的主对角线元素增长为原来的$k$倍,用Jacobi迭代法计算,迭代次数及迭代矩阵的谱半径如\Cref{tab:ex3_a_diag}所示。

\begin{table}
    \centering
    \caption{将$\boldsymbol{A}$的主对角线元素增长为原来的$k$倍后,给定迭代误差$10^{-5}$时,Jacobi迭代法的迭代次数$m$及迭代矩阵$\boldsymbol{B}$的谱半径$\rho(\boldsymbol{B})$。}
    \label{tab:ex3_a_diag}
    \begin{tabular}{c|cccccccccc}
        \toprule
        \(k\) & 1 & 2 & 4 & 8 & 16 & 32 & 64 & 128 & 256 & 512\tabularnewline
        \midrule
        \(m\) & 15 & 9 & 7 & 6 & 5 & 4 & 4 & 4 & 3 & 3\tabularnewline
        \(\rho(\boldsymbol{B})\) & 0.489 & 0.245 & 0.122 & 0.061 & 0.031 & 0.015
        & 0.008 & 0.004 & 0.002 & 0.001\tabularnewline
        \bottomrule
    \end{tabular}
\end{table}

\subsubsection{结果分析}

在不同的初始向量和右端项下,Jacobi和Gauss-Seidel迭代法产生的迭代向量序列均收敛,相比于Jacobi迭代法,Gauss-Seidel迭代法的迭代次数更少,收敛速度更快。这是由迭代矩阵$\boldsymbol{B}$的谱半径决定的,通过计算得出,Jacobi的迭代矩阵$\boldsymbol{B}$的谱半径$\rho(\boldsymbol{B}) = 0.4893$,Gauss-Seidel的迭代矩阵$\boldsymbol{B}$的谱半径$\rho(\boldsymbol{B}) = 0.2523$,它们都小于1,因此两种迭代法均收敛,由于Gauss-Seidel的迭代矩阵$\boldsymbol{B}$的谱半径小于Jacobi的迭代矩阵,因此Gauss-Seidel迭代法的收敛速度更快。从另一个角度来看,由于$\boldsymbol{A}$是严格对角占优的,因此也可以判断出两种迭代法均收敛。

随着$\boldsymbol{A}$主对角线元素的成倍增长,Jacobi迭代法的迭代次数越来越少,收敛速度越来越快。这是因为当$\boldsymbol{A}$的主对角线元素增长时,$\boldsymbol{A}$的对角线更占优,从\Cref{tab:ex3_a_diag}可以看出,其对应的迭代矩阵$\boldsymbol{B}$的谱半径更小,因此收敛速度更快。

\subsubsection{结论}

若线性方程组的系数矩阵是严格对角占优的,则Jacobi和Gauss-Seidel两种迭代法的迭代序列均收敛,一般来说,Gauss-Seidel迭代法比Jacobi迭代法的收敛速度更快。

\subsection{Chap5-Ex9 种群(计算题)}

\subsubsection{算法设计}

根据题意,种群年龄为$k=1,2,...,n$,当年年龄$k$的种群数量记作$x_k$,繁殖率记作$b_k$,自然存活率记作$s_k$,收获量记作$h_k$,来年年龄$k$的种群数量为$\tilde{x}_k$,稳定种群内,有$\tilde{x}_k=x_k$。

\paragraph{矩阵模型} 已知$b_k,s_k,h_k$时,在稳定种群内$\tilde{x}_k = x_k$,因此有,
\begin{equation}\label{eq:ex9_scalar}
    x_1 = \sum_{k=1}^n b_k x_k, \quad x_{k+1}=s_k x_k - h_k,\quad k=1,2,...,n-1
\end{equation}
记$\boldsymbol{x} = (x_1, x_2, ..., x_n)^T$,$\boldsymbol{h}=(0, h_1, h_2, ..., h_{n-1})^T$,记Leslie矩阵为,
\begin{equation}
    \boldsymbol{L} = \left(
    \begin{matrix}
        b_1     & b_2       & \cdots & \cdots & b_n \\
        s_1     & 0         & \cdots & \cdots & 0   \\
        0       & s_2       & \ddots &        & 0   \\
        \vdots  &           & \ddots & \ddots & \vdots \\
        0       & \cdots    & 0      & s_{n-1}& 0 \\
    \end{matrix}
    \right)
\end{equation}
则可将\Cref{eq:ex9_scalar}表示为矩阵形式,
\begin{equation}
    \boldsymbol{x} = \boldsymbol{Lx} - \boldsymbol{h}
\end{equation}
化简得,
\begin{equation}\label{eq:ex9_model}
    (\boldsymbol{L} - \boldsymbol{I})\boldsymbol{x} = \boldsymbol{h}
\end{equation}
即为稳定种群数量的矩阵模型。

\paragraph{数值求解} 利用Matlab自带的左除法,即可求出\Cref{eq:ex9_model}的数值解。

\subsubsection{Matlab程序}

请参见附录\ref{sec:ex9_code}。

\subsubsection{计算结果}

由题目数据,可确定Leslie矩阵,
\begin{equation}\label{eq:ex9_matrix_a}
    \boldsymbol{L} = \left(\begin{matrix}
        0   & 0   & 5   & 3   & 0\\
        0.4 & 0   & 0   & 0   & 0\\
        0   & 0.6 & 0   & 0   & 0\\
        0   & 0   & 0.6 & 0   & 0\\
        0   & 0   & 0   & 0.4 & 0\\
    \end{matrix}\right)
\end{equation}

当$\boldsymbol{h} = (0,500,400,200,100)^T$时,计算得出,
\begin{equation}\label{eq:ex9_normal}
    \boldsymbol{x} = (8481,2892,1335,601,141)^T
\end{equation}

当$\boldsymbol{h}=(0,500,500,500,500)^T$时,计算得出,
\begin{equation}\label{eq:ex9_negative}
    \boldsymbol{x} = (10981,3892,1835,601,-259)^T
\end{equation}
然而,种群数量$x_k$不可能为负数。产生负数的原因可能是收获量过大,供不应求,打破了种群的平衡状态。因此,为了能达到给定的收获量,需要采取相应的措施来维持生态平衡。

一种可能的措施是通过改善种群的生活条件,提高种群的存活率。考虑到$x_5$出现负数,应当尽可能提高$x_4$的存活率$s_4$从而增加$x_5$的数量,如果将$s_4$提高到0.9,则计算得出,
\begin{equation}\label{eq:ex9_survival}
    \boldsymbol{x} = (10981,3892,1835,601,41)^T
\end{equation}
达到了种群数量的平衡。

另一种可能的措施是通过增加种群规模,调节种群的繁殖率,从而提高种群的稳定性。如果将$b_3$调节为3,则计算得出,
\begin{equation}\label{eq:ex9_reproduction}
    \boldsymbol{x} = (35132,13553,7632,4079,1132)^T
\end{equation}
也达到了种群数量的平衡,此时种群规模更大,平衡性更强。

当然,两种方式可结合使用,使种群更稳定,满足收获量的需求。

\subsubsection{结果分析}

\paragraph{误差分析} 记种群数量的计算值为$\boldsymbol{x}^*$,收获量的真实值为$\boldsymbol{h}$,则绝对误差为$\boldsymbol{e} = (\boldsymbol{L} - \boldsymbol{I})\boldsymbol{x}^* - \boldsymbol{h}$,通过计算得出各结果的绝对误差,如\Cref{tab:ex9_error}所示,其中$e_i$为$\boldsymbol{e}$的第$i$个分量。可以看出,所有计算结果的绝对误差均控制在$10^{-11}$以内,结果有效。

\begin{table}
    \centering
    \caption{各年龄种群数量计算结果的绝对误差}
    \label{tab:ex9_error}
    \begin{tabular}{c|ccccc}
        \toprule
        绝对误差 (\(\times 10^{-12}\)) & \(e_1\) & \(e_2\) &
        \(e_3\) & \(e_4\) & \(e_5\)\tabularnewline
        \midrule
        结果\ref{eq:ex9_normal} & -1.023 & 0.455 & 0.227 & 0.000 & -0.028\tabularnewline
        结果\ref{eq:ex9_negative} & -1.023 & 0.455 & 0.000 & -0.114 & 0.000\tabularnewline
        结果\ref{eq:ex9_survival} & -1.023 & 0.455 & 0.000 & -0.114 & 0.000\tabularnewline
        结果\ref{eq:ex9_reproduction} & 0.000 & -1.819 & 0.000 & 0.000 & 0.000\tabularnewline
        \bottomrule
    \end{tabular}
\end{table}

\paragraph{稳定性分析} 通过计算得出,由\Cref{eq:ex9_matrix_a}确定的矩阵$\boldsymbol{L} - \boldsymbol{I}$的条件数为87.19,病态程度一般,模型的稳定性一般。

\subsubsection{结论}

若要求种群内年龄为$1\sim 5$的收获量分别为500,400,200,100,100,则稳定种群内年龄为$1\sim 5$的个体数量分别为8481,2892,1335,601,141。要使种群各年龄收获量均为500,则可以通过提高存活率,调节繁殖率以达到目的。

\section{收获与建议}

在本次实验中,我通过使用Matlab,掌握了数值求解线性代数方程组的方法,对迭代法的收敛性和解的稳定性有了更深的理解,用线性代数方程组解决了简化的实际问题,在解决实际问题的过程中,我对数学方法的原理和应用有了更深刻的理解。

希望助教能对每次的实验进行详细的解答,希望老师在未来的课堂上介绍更多数学应用的前沿知识。

\section{附录:Matlab程序代码}

\subsection{Chap5-Ex1}\label{sec:ex1_code}

\lstinputlisting[language=Matlab]{../src/ex1.m}

\subsection{Chap5-Ex3}\label{sec:ex3_code}

\lstinputlisting[language=Matlab]{../src/ex3.m}

\subsection{Chap5-Ex9}\label{sec:ex9_code}

\lstinputlisting[language=Matlab]{../src/ex9.m}

\section{附录:详细计算结果}

\subsection{Chap5-Ex1}\label{sec:ex1_details}

\begin{table}[h]
    \centering
    \caption{在$\boldsymbol{A}_1(n,n)$处加扰动后的$\boldsymbol{x}_1$求解结果}
    \label{tab:ex1_a1}
    \begin{tabular}{l|l}
        \toprule
        \((n,\varepsilon)\) & \(\boldsymbol{x}_1^T\)\tabularnewline
        \midrule
        \((5,10^{-10})\) & \(1.0000,1.0000,1.0000,1.0000,1.0000\)\tabularnewline
        \((5,10^{-8})\) & \(1.0000,1.0000,1.0000,1.0000,1.0000\)\tabularnewline
        \((5,10^{-6})\) & \(0.9993,1.0025,0.9967,1.0019,0.9996\)\tabularnewline
        \((7,10^{-10})\) &
        \(1.0000,1.0000,1.0000,1.0000,1.0000,1.0000,1.0000\)\tabularnewline
        \((7,10^{-8})\) &
        \(0.9999,1.0002,0.9995,1.0005,0.9997,1.0001,1.0000\)\tabularnewline
        \((7,10^{-6})\) &
        \(0.9950,1.0245,0.9503,1.0536,0.9676,1.0104,0.9986\)\tabularnewline
        \((9,10^{-10})\) &
        \(1.0000,1.0000,1.0000,1.0001,0.9999,1.0000,1.0000,1.0000,1.0000\)\tabularnewline
        \((9,10^{-8})\) &
        \(0.9998,1.0015,0.9961,1.0060,0.9944,1.0034,0.9987,1.0003,1.0000\)\tabularnewline
        \((9,10^{-6})\) &
        \(0.9758,1.1481,0.6062,1.5958,0.4389,1.3367,0.8743,1.0267,0.9975\)\tabularnewline
        \bottomrule
    \end{tabular}
\end{table}

\begin{table}[h]
    \centering
    \caption{在$\boldsymbol{A}_2(n,n)$处加扰动后的$\boldsymbol{x}_2$求解结果}
    \label{tab:ex1_a2}
    \begin{tabular}{l|l}
        \toprule
        \((n,\varepsilon)\) & \(\boldsymbol{x}_2^T\)\tabularnewline
        \midrule
        \((5,10^{-10})\) & \(1.0000,1.0000,1.0000,1.0000,1.0000\)\tabularnewline
        \((5,10^{-8})\) & \(1.0000,1.0001,0.9994,1.0009,0.9996\)\tabularnewline
        \((5,10^{-6})\) & \(0.9994,1.0121,0.9457,1.0845,0.9578\)\tabularnewline
        \((7,10^{-10})\) &
        \(1.0000,1.0001,0.9995,1.0020,0.9962,1.0033,0.9989\)\tabularnewline
        \((7,10^{-8})\) &
        \(0.9999,1.0045,0.9546,1.1816,0.6594,1.2997,0.9001\)\tabularnewline
        \((7,10^{-6})\) &
        \(0.9990,1.0417,0.5830,2.6679,-2.1273,3.7520,0.0827\)\tabularnewline
        \((9,10^{-10})\) &
        \(1.0000,1.0012,0.9785,1.1577,0.4085,2.2304,-0.4355,1.8789,0.7803\)\tabularnewline
        \((9,10^{-8})\) &
        \(0.9999,1.0054,0.9055,1.6933,-1.6000,6.4079,-5.3093,4.8628,0.0343\)\tabularnewline
        \((9,10^{-6})\) &
        \(0.9999,1.0056,0.9021,1.7177,-1.6914,6.5980,-5.5310,4.9986,0.0004\)\tabularnewline
        \bottomrule
    \end{tabular}
\end{table}

\begin{table}[h]
    \centering
    \caption{在$\boldsymbol{b}_1(n)$处加扰动后的$\boldsymbol{x}_1$求解结果}
    \label{tab:ex1_b1}
    \begin{tabular}{l|l}
        \toprule
        \((n,\varepsilon)\) & \(\boldsymbol{x}_1^T\)\tabularnewline
        \midrule
        \((5,10^{-10})\) & \(1.0000,1.0000,1.0000,1.0000,1.0000\)\tabularnewline
        \((5,10^{-8})\) & \(1.0000,1.0000,1.0000,1.0000,1.0000\)\tabularnewline
        \((5,10^{-6})\) & \(1.0007,0.9975,1.0033,0.9981,1.0004\)\tabularnewline
        \((7,10^{-10})\) &
        \(1.0000,1.0000,1.0000,1.0000,1.0000,1.0000,1.0000\)\tabularnewline
        \((7,10^{-8})\) &
        \(1.0001,0.9998,1.0005,0.9995,1.0003,0.9999,1.0000\)\tabularnewline
        \((7,10^{-6})\) &
        \(1.0050,0.9755,1.0497,0.9464,1.0324,0.9896,1.0014\)\tabularnewline
        \((9,10^{-10})\) &
        \(1.0000,1.0000,1.0000,0.9999,1.0001,1.0000,1.0000,1.0000,1.0000\)\tabularnewline
        \((9,10^{-8})\) &
        \(1.0002,0.9985,1.0039,0.9940,1.0056,0.9966,1.0013,0.9997,1.0000\)\tabularnewline
        \((9,10^{-6})\) &
        \(1.0243,0.8516,1.3948,0.4027,1.5624,0.6625,1.1260,0.9732,1.0025\)\tabularnewline
        \bottomrule
    \end{tabular}
\end{table}

\begin{table}[h]
    \centering
    \caption{在$\boldsymbol{b}_2(n)$处加扰动后的$\boldsymbol{x}_2$求解结果}
    \label{tab:ex1_b2}
    \begin{tabular}{l|l}
        \toprule
        \((n,\varepsilon)\) & \(\boldsymbol{x}_2^T\)\tabularnewline
        \midrule
        \((5,10^{-10})\) & \(1.0000,1.0000,1.0000,1.0000,1.0000\)\tabularnewline
        \((5,10^{-8})\) & \(1.0000,0.9999,1.0006,0.9991,1.0004\)\tabularnewline
        \((5,10^{-6})\) & \(1.0006,0.9874,1.0567,0.9118,1.0441\)\tabularnewline
        \((7,10^{-10})\) &
        \(1.0000,0.9999,1.0005,0.9980,1.0038,0.9967,1.0011\)\tabularnewline
        \((7,10^{-8})\) &
        \(1.0001,0.9950,1.0505,0.7982,1.3784,0.6670,1.1110\)\tabularnewline
        \((7,10^{-6})\) &
        \(1.0120,0.4955,6.0450,-19.1802,38.8378,-32.2973,12.0991\)\tabularnewline
        \((9,10^{-10})\) &
        \(1.0000,0.9984,1.0276,0.7978,1.7581,-0.5768,2.8397,-0.1263,1.2816\)\tabularnewline
        \((9,10^{-8})\) &
        \(1.00,0.84,3.76,-19.22,76.81,-156.69,184.97,-111.63,29.16\)\tabularnewline
        \((9,10^{-6})\) &
        \(1,-15,277,-2021,7582,-15768,18398,-11262,2817\)\tabularnewline
        \bottomrule
    \end{tabular}
\end{table}

\end{document}