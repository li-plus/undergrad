% (钢管下料)某钢管零售商从钢管厂进货,将钢管按照顾客的要求切割后售出。从钢管厂进货时得到的原料钢管长度都是 1850 毫米。现有一客 户需要 15 根 290 毫米长、 28 根 315 毫米长、 21 根 350 毫米长和 30 根 455 毫米长的钢管。为了简化生产过程,规定所使用的切割模式的种类不能超过 4 种,使用频率最高的一种切割模式按照一根原料钢管价值的 1/10 增加费用,使用频率次之的切割模式按照一根原料钢管价值的 2/10 增加费用,依次类推,且每种切割模式下的切割次数不能太多(一根原料钢管最多生产 5 根产品)。此外,为了减少余料浪费,每种切割模式下的余料浪费不能超过 100 毫米。为了使总费用最小,应如何下料。

\subsubsection{问题分析}

题目设置了一个钢管生产场景,给出了原料钢管的长度,不同长度钢管的需求量,切割模式的成本,切割钢管的数量限制,需要确定最优生产方案,使得总费用最小。题目构成了一个钢管下料问题,这是一个经典的整数规划问题。

\subsubsection{模型假设}

为了简化实际情况,模型基于以下假设,
\begin{enumerate}
    \item 切割过程中没有物料损失,能够精准控制钢管长度,不产生次品。
    \item 生产余料的价值为零。
    \item 没有仓储和运输费用。
\end{enumerate}

\subsubsection{模型建立}

设用户需要$m$种规格的钢管,第$i$种规格的钢管长度为$d_i$,需求量为$c_i$,设一共采用$n$种切割模式,在第$j$种切割模式下,每根原料钢管的处理成本为$p_j$,共切割$x_j$根原料钢管,生产长度为$d_i$的钢管数量为$r_{ij}$,其中$i=1,2,\cdots,m$,$j=1,2,\cdots,n$。记原料钢管长度为$Q$,每种切割模式下余料的最大长度为$q$,每根原料钢管最多生产$k$根产品。

为方便叙述,记钢管长度向量$\mathbf{d} = (d_1,\cdots,d_m)$,需求向量$\mathbf{c} = (c_1,\cdots,c_m)$,生产矩阵$\mathbf{R} = (r_{ij})_{m \times n}$,原料消耗向量$\mathbf{x} = (x_1, \cdots, x_n)$,成本系数向量$\mathbf{p} = (p_1,\cdots,p_n)$。

为了使切割模式与价格对应,这里指定大小顺序为,
\begin{equation}\label{eq:ex11_cons_order}
    x_{j+1} \le x_j, \quad j=1,2,\cdots,n-1
\end{equation}

生产需要满足客户需求,注意这里的$\ge$符号表示按分量比较,
\begin{equation}\label{eq:ex11_cons_demand}
    \mathbf{Rx} \ge \mathbf{c}
\end{equation}

一根原料钢管最多生产$k$根产品,
\begin{equation}
    \sum_{i=1}^m r_{ij} \le k, \quad j=1,2,\cdots,n
\end{equation}

不同切割模式下需要满足余料限制,
\begin{equation}\label{eq:ex11_cons_remain}
    Q - q \le \mathbf{R}^T \mathbf{d} \le Q
\end{equation}

再加上非负整数约束,
\begin{equation}\label{eq:ex11_cons_int}
    \mathbf{x} \in \mathbb{N}^{m}, \quad \mathbf{R} \in \mathbb{N}^{m \times n}
\end{equation}

在此基础上,设每根原料钢管的采购成本为单位成本,需要最小化生产总费用$f$,
\begin{equation}\label{eq:ex11_objective}
    \min f = \mathbf{px}
\end{equation}

这是一个整数规划模型,目标函数为\Cref{eq:ex11_objective},决策变量为$\mathbf{R}$和$\mathbf{x}$,在约束条件\Cref{eq:ex11_cons_order},\Cref{eq:ex11_cons_demand},\Cref{eq:ex11_cons_remain}和\Cref{eq:ex11_cons_int}。

\subsubsection{算法设计}

对于整数规划,可以采用LINGO软件求解,需要使用\texttt{@gin}命令将决策变量限制在整数域内。

\subsubsection{程序}

请参见附录\ref{sec:ex11_code}。

\subsubsection{计算结果}

LINGO将该问题识别为PIQP (Pure Integer Quadratic Program),经过3,500,150次迭代,求得全局最优解,总费用$f$的最小值为21.5倍单位成本,决策变量$\mathbf{R}$和$\mathbf{x}$的最优值为,
\begin{equation}
    \mathbf{R} = \left(\begin{matrix}
        1 & 0 & 2 & 0\\
        2 & 0 & 0 & 0\\
        0 & 5 & 1 & 0\\
        2 & 0 & 2 & 4
    \end{matrix}\right)
    ,\quad
    \mathbf{x} = \left(\begin{matrix}
        14\\
        4\\
        1\\
        0
    \end{matrix}\right)
\end{equation}

将上述结果进行整理,得到具体切割模式及原料钢管消耗数量,如\Cref{tab:ex11_result}所示。注意到第四种切割模式的生产量为零,因此该切割模式无意义,应当将其省略。

\begin{table}[H]
    \centering
    \caption{具体切割模式及原料钢管消耗数量}
    \label{tab:ex11_result}
    \begin{tabular}{c|ccccc|c}
        \toprule
        & 290 mm & 315 mm & 350 mm & 455 mm & 余料 (mm) &
        原料钢管\tabularnewline
        \midrule
        切割模式1 & 1 & 2 & 0 & 2 & 20 & 14\tabularnewline
        切割模式2 & 0 & 0 & 5 & 0 & 100 & 4\tabularnewline
        切割模式3 & 2 & 0 & 1 & 2 & 10 & 1\tabularnewline
        \bottomrule
    \end{tabular}
\end{table}

\subsubsection{结果的数学分析}

整数规划是一个NPC问题,在求解过程中,往往需要通过增加约束条件,使得分支定界法能够及时剪枝,从而加快求解速度。额外的约束可以根据常理人为添加,也可以通过割平面算法求得。

在本题中,原料消耗量大小顺序约束\Cref{eq:ex11_cons_order}其实是不必要的,只要指定了成本系数向量$\mathbf{p}$,那么在最优解中,产量最大的切割模式必定对应最低的成本系数,否则,通过交换两种切割模式的顺序,就可以得到更低的成本。然而,如果将该约束去掉,则LINGO需要11,089,340次迭代才能求解出相同的结果,求解速度大约下降到了原来的1/4。

反过来,如果增加一个约束会怎么样呢?考虑到一根1850毫米长原料钢管的余料最多为100毫米,即使全部生产最长的455毫米长钢管,也至少生产4根才能满足余料约束,因此有,
\begin{equation}
    \sum_{i=1}^m r_{ij} \ge 4, \quad j=1,2,\cdots,n
\end{equation}

加上这个约束后,LINGO只需要828,566次迭代就能求出同样的结果,求解速度加快到了原来的4倍,因此附录的源码也增加了这个约束。

可见,在不改变最优解的情况下,约束越强,求解速度越快。

\subsubsection{结果的实际意义}

该计算结果具有一定的实用价值,可作为制定生产方案的重要参考。然而,该模型仍相对简单,在实际应用中,还应当综合考虑工厂的实际情况,例如原料钢管的运输费用,剩余成品的仓储成本,切割过程的物料损失,产品的次品率,余料的利用价值等因素,才能制定出合适的生产方案。

\subsubsection{结论}

应当使用三种切割模式。

第一种切割模式处理原料钢管14根,每根原料钢管切割成1根290毫米长,2根315毫米长和2根455毫米长钢管,余料为20毫米长。

第二种切割模式处理原料钢管4根,每根原料钢管切割成5根350毫米长钢管,余料为100毫米长。

第三种切割模式处理原料钢管1根,每根原料钢管切割成2根290毫米长,1根350毫米长和2根455毫米长钢管,余料为10毫米长。

此时总费用最小,为单根原料钢管采购成本的21.5倍。
