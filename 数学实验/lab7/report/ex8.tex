% 8. 某储蓄所每天的营业时间是上午9 时到下午5 时。根据经验,每天不同时间段所需要的服务员数量如表10.7。
% 时间段(时) 9~10 10~11 11~12 12~1 1~2 2~3 3~4 4~5
% 服务员数量 4 3 4 6 5 6 8 8
% 储蓄所可以雇佣全时和半时两类服务员。全时服务员每天报酬100 元,从上午9 时到下午5 时工作,但中午12 时到下午2 时之间必须安排1 小时的午餐时间。储蓄所每天可以雇佣不超过3名的半时服务员,每个半时服务员必须连续工作4 小时,报酬40 元。问该储蓄所应如何雇佣全时和半时两类服务员。如果不能雇佣半时服务员,每天至少增加多少费用。如果雇佣半时服务员的数量没有限制,每天可以减少多少费用。

\subsubsection{问题分析}

题目给定储蓄所各时段的服务员数量要求,全时服务员和半时服务员的每日报酬和工作时间,需要确定聘用方案,这是一个整数规划问题。

\subsubsection{模型假设}

为了简化实际情况,模型基于以下假设,
\begin{enumerate}
    \item 总收益与服务员数量无关。
    \item 各时段的人流分布稳定,未来的人流分布与经验数据相差不大。
\end{enumerate}

\subsubsection{模型建立}

\paragraph{半时服务员不超过3名时} 在全时服务员中,设12时到1时午餐的人数为$x_1$,1时到2时午餐的人数为$x_2$,在半时服务员中,设9时到1时工作的人数为$y_1$,10时到2时工作的人数为$y_2$,11时到3时工作的人数为$y_3$,12时到4时工作的人数为$y_4$,1时到5时工作的人数为$y_5$,则各种服务员在各时段的工作时间表如\Cref{tab:ex8_task}所示。

\begin{table}[H]
    \centering
    \caption{各种服务员在各时段的工作时间表}
    \label{tab:ex8_task}
    \begin{tabular}{c|cccccccc}
        \toprule
        & $9\sim 10$ & $10\sim 11$ & $11\sim 12$ & $12\sim 1$ & $1\sim 2$ & $2\sim 3$ & $3\sim 4$ & $4\sim 5$\tabularnewline
        \midrule
        \(x_1\) & \cmark & \cmark &
        \cmark & & \cmark & \cmark
        & \cmark & \cmark\tabularnewline
        \(x_2\) & \cmark & \cmark &
        \cmark & \cmark & & \cmark
        & \cmark & \cmark\tabularnewline
        \(y_1\) & \cmark & \cmark &
        \cmark & \cmark & & & &\tabularnewline
        \(y_2\) & & \cmark & \cmark &
        \cmark & \cmark & & &\tabularnewline
        \(y_3\) & & & \cmark & \cmark &
        \cmark & \cmark & &\tabularnewline
        \(y_4\) & & & & \cmark & \cmark &
        \cmark & \cmark &\tabularnewline
        \(y_5\) & & & & & \cmark & \cmark &
        \cmark & \cmark\tabularnewline
        \bottomrule
    \end{tabular}
\end{table}

将工作时间表记为矩阵$\mathbf{A} \in \mathbb{R}^{7 \times 8}$,其中符号{\cmark}对应数值1,空白位置对应数值0。记各种服务员的聘用人数为$\mathbf{x} = (x_1, x_2, y_1, y_2, y_3, y_4, y_5)^T$,各时段服务员需求量为$\mathbf{b} = (4, 3, 4, 6, 5, 6, 8, 8)^T$,则需求量约束可表示如下,注意这里的$\ge$符号表示按分量比较,
\begin{equation}\label{eq:ex8_cons_demand}
    \mathbf{A}^T \mathbf{x} \ge \mathbf{b}
\end{equation}

半时服务员的最大数量约束为,
\begin{equation}\label{eq:ex8_cons_parttime}
    \sum_{i=1}^5 y_i \le 3
\end{equation}

还需要加上非负整数约束,
\begin{equation}\label{eq:ex8_cons_int}
    \mathbf{x} \in \mathbb{N}^7
\end{equation}

在此基础上,需要最小化每日人力成本$f$,
\begin{equation}\label{eq:ex8_obj}
    \min f = 100(x_1 + x_2) + 40(y_1 + y_2 + y_3 + y_4 + y_5)
\end{equation}

这就构成了一个整数规划模型,目标函数为\Cref{eq:ex8_obj},决策变量为$\mathbf{x}$,约束条件为\Cref{eq:ex8_cons_demand},\Cref{eq:ex8_cons_parttime}和\Cref{eq:ex8_cons_int}。

\paragraph{不能雇佣半时服务员时} 只需把半时服务员的约束条件\Cref{eq:ex8_cons_parttime}修改为,
\begin{equation}\label{eq:ex8_cons_no_parttime}
    y_i = 0, \quad i=1,2,3,4,5
\end{equation}

此时,目标函数为\Cref{eq:ex8_obj},决策变量为$\mathbf{x}$,约束条件为\Cref{eq:ex8_cons_demand},\Cref{eq:ex8_cons_no_parttime}和\Cref{eq:ex8_cons_int}。

\paragraph{半时服务员数量没有限制时} 只需把半时服务员的约束条件\Cref{eq:ex8_cons_parttime}去掉即可。此时,目标函数为\Cref{eq:ex8_obj},决策变量为$\mathbf{x}$,约束条件为\Cref{eq:ex8_cons_demand}和\Cref{eq:ex8_cons_int}。

\subsubsection{算法设计}

对于该整数规划模型,可采用LINGO求解,对应的LINGO模型类别为PILP (Pure Integer Linear Program),求解方法为分支定界法 (B-and-B)。

\subsubsection{程序}

请参见附录\ref{sec:ex8_code}。

\subsubsection{计算结果}

\paragraph{半时服务员不超过3名时} LINGO经过19次迭代,求得全局最优解,得到总人力成本$f$的最小值为820元,各决策变量的最优值为,
\begin{equation}
    x_1 = 3, \quad x_2 = 4, \quad y_1 = 0, \quad y_2 = 2, \quad y_3 = 0, \quad y_4 = 0, \quad y_5 = 1
\end{equation}

\paragraph{不能雇佣半时服务员时} LINGO经过0次迭代,求得全局最优解,得到总人力成本$f$的最小值为1100元,各决策变量的最优值为,
\begin{equation}
    x_1 = 5, \quad x_2 = 6, \quad y_1 = 0, \quad y_2 = 0, \quad y_3 = 0, \quad y_4 = 0, \quad y_5 = 0
\end{equation}

此时,相比于默认情况,每天需要增加的费用为280元。

\paragraph{半时服务员数量没有限制时} LINGO经过2次迭代,求得全局最优解,得到总人力成本$f$的最小值为560元,各决策变量的最优值为,
\begin{equation}
    x_1 = 0, \quad x_2 = 0, \quad y_1 = 6, \quad y_2 = 0, \quad y_3 = 0, \quad y_4 = 0, \quad y_5 = 8
\end{equation}

此时,相比于默认情况,每天可以减少的费用为260元。

\subsubsection{结果的数学分析}

从计算结果可以看出,可招聘的半时服务员的数量越多,总费用就越低,这是因为半时服务员的时薪比全时服务员更低,在合理的安排下,招聘越多半时服务员,越有利于降低总成本。

此外,在本题条件下,全局最优解不唯一,例如当半时服务员不超过3名时,以下也是一个全局最优解,每日最小费用同样为820元。
\begin{equation}
    x_1 = 3, \quad x_2 = 4, \quad y_1 = 0, \quad y_2 = 0, \quad y_3 = 2, \quad y_4 = 0, \quad y_5 = 1
\end{equation}

\subsubsection{结果的实际意义}

该计算结果具有一定的实际意义,可作为制定招聘方案的重要参考。在实际情况下,还需考虑节假日与工作日的服务员数量差异,服务人数及总收益与服务员数量的关系,以及人流分布的波动情况,根据实际情况对模型进行微调。

\subsubsection{结论}

半时服务员不超过3名时,储蓄所应当聘请7名全时服务员,其中3名在12时到1时午餐,另外4名在1时到2时午餐,还应当聘请3名半时服务员,其中2名在10时到2时工作,另外1名在1时到5时工作,此时每日总费用最低,为820元。

不能雇佣半时服务员时,储蓄所应当聘请11名全时服务员,其中5名在12时到1时午餐,另外6名在1时到2时午餐,此时每日总费用最低,为1100元。相比于默认情况,每天需要增加的费用为280元。

半时服务员数量没有限制时,储蓄所应当聘请14名半时服务员,其中6名在9时到1时工作,另外8名在1时到5时工作,此时每日总费用最低,为560元。相比于默认情况,每天可以减少的费用为260元。
