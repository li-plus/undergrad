% 5. 甲方向乙方成批供货,甲方承诺合格率为90\%,双方商定置信概率为95\%。现从一批货中抽取50 件,43 件为合格品,问乙方应否接受这批货物? 你能为乙方不接受它出谋划策吗?

\subsubsection{问题分析}

题目给出了产品承诺的合格率和置信概率,以及一次抽检的结果,需要确定该产品是否达标,这是一个0-1分布总体均值的假设检验问题。

\subsubsection{模型假设}

为了简化实际情况,模型基于以下假设,
\begin{enumerate}
    \item 每件产品是否合格服从0-1分布。
    \item 每批货物的数量充分大。
\end{enumerate}

\subsubsection{模型建立}

根据中心极限定理,当样本量充分大时,样本的均值近似服从正态分布。根据假设,每批货物的数量充分大,因此可对这批货物总体的合格率$p$做正态总体均值的假设检验,记甲方承诺的合格率为$p_0$,考虑到对合格率的检查应当采用单侧检验,则原假设$H_0$和对立假设$H_1$为,
\begin{equation}
    H_0: p \ge p_0, \quad H_1: p < p_0
\end{equation}

对上述假设进行$z$检验,记样本容量为$n$,合格率为$\overline{x}$,显著性水平为$\alpha$,$N(0,1)$的$\alpha$分位数为$u_\alpha$,并且,
\begin{equation}
    z = \frac{\overline{x} - p_0}{\sqrt{p_0(1-p_0)/n}}
\end{equation}

则当$z \ge u_\alpha$时,$H_0$被接受,乙方应当接受这批货物;否则,$H_0$被拒绝,乙方不应当接受这批货物。

从上述理论分析可以看出,作为乙方,如果不想接受这批货物,即拒绝$H_0$,有以下几种可能的策略。
\begin{enumerate}
    \item 与甲方商定新的置信概率$1-\alpha$。
    \item 要求甲方调节产品合格率$p_0$。
    \item 若样本合格率$\overline{x}$保持不变,可调节样本容量$n$。
\end{enumerate}

\subsubsection{算法设计}

对于$z$检验,可采用\texttt{ztest}命令。

\subsubsection{程序}

请参见附录\ref{sec:ex5_code}。

\subsubsection{计算结果}

在双方商定的95\%置信概率下,经过$z$检测,原假设被接受,得到$p$值为0.1729,置信区间为$(-\infty,0.9298]$,因此,乙方应当接受这批货物。

作为乙方,如果不想接受这批货物,可以采取以下其中一种策略。
\begin{enumerate}
    \item 与甲方商定新的置信概率,不得超过82.71\%。
    \item 要求甲方提高产品合格率,不得低于92.23\%。
    \item 若样本合格率保持不变,可以增大抽检样本容量,不得低于154。
\end{enumerate}

\subsubsection{结果的数学分析}

对于置信概率的调节,可以利用已经求得的$p$值,使置信概率低于$1-p$,此时显著性水平$\alpha$将高于$p$值,原假设将被拒绝。

对于产品合格率和抽检样本容量的调节,需要在其它变量固定不变时,求解方程$z < u_\alpha$。

\subsubsection{结果的实际意义}

该计算结果具有一定的实际意义,可作为乙方制定采购策略的重要参考。在实际应用中,只要保证随机抽样,且样本量足够大,就应当按照商定好的置信概率行事,这样对双方都是有利的。

如果抽检样本的合格率低于产品合格率,乙方应当接受但实在不愿意接受时,应当采取第3种策略来进一步验证,即加大抽检量,此时一般有两种情况:第一,这批货物的总体合格率是符合承诺标准的,那么随着抽检量的增加,抽检样本的合格率应该逐步接近或高于承诺的合格率,原假设始终被接受,乙方应当接受这批货物;第二,这批货物的总体合格率不符合标准,那么随着抽检量的增加,原假设必然在某一时刻被拒绝,乙方可以不接受这批货物。

\subsubsection{结论}

乙方应当接受这批货物。若乙方不想接受,可以与甲方商定,将置信概率降低到82.71\%以下,或者将合格率提高到92.23\%以上,或者将抽检样本容量提高到154以上。
