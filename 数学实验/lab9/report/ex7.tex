\subsubsection{问题分析}

题目设置了一个对照实验,分为胃溃疡病人和正常人两组,每组给出了30人的溶菌酶含量数据,需要确定病人和正常人的溶菌酶含量有无显著差别,这是一个两总体均值的假设检验问题。

\subsubsection{模型假设}

为了简化实际情况,模型基于以下假设,
\begin{enumerate}
    \item 正常人和胃溃疡病人的溶菌酶含量均服从正态分布。
    \item 除特别说明外,原始数据均准确无误。
\end{enumerate}

\subsubsection{模型建立}

由于样本量较小,未能通过总体分布的正态性检验,但是根据生活经验,很多医学现象都是服从正态分布的,因此这里假设病人和正常人的溶菌酶含量两个总体均服从正态分布,病人的溶菌酶含量服从$N(\mu_1, \sigma_1^2)$,正常人的溶菌酶含量服从$N(\mu_2, \sigma_2^2)$,使用两总体均值的假设检验方法,记原假设$H_0$和对立假设$H_1$为,
\begin{equation}
    H_0: \mu_1 = \mu_2, \quad H_1: \mu_1 \ne \mu_2
\end{equation}

由于总体方差未知,因此采用$t$检验。若原假设$H_0$被接受,则病人与正常人的溶菌酶含量无显著差别;若被拒绝,则有显著差别。

\subsubsection{算法设计}

在MATLAB的实现中,可采用\texttt{ttest2}命令进行两总体均值的假设检验。

\subsubsection{程序}

请参见附录\ref{sec:ex7_code}。

\subsubsection{计算结果}

使用病人组的全部数据时,用点估计求得总体分布为$N(15.33,16.45^2)$。经过$t$检验,得到$p$值为0.0251,低于默认的显著性水平0.05,原假设被拒绝,表明病人与正常人的溶菌酶含量有显著差别。

去掉病人组的最后5个数据后,用点估计求得总体分布为$N(11.51,11.75^2)$。经过$t$检验,得到$p$值为0.1558,高于默认的显著性水平0.05,原假设被接受,表明病人与正常人的溶菌酶含量无显著差别。

\subsubsection{结果的数学分析}

经过计算,病人组的最后5个数据很可能有误,因为其中包含了2个$3\sigma$之外,1个$2\sigma$之外的离群点,使参数估计产生了较大误差。为了减小误差,应当尽可能降低原始数据的噪声。

\subsubsection{结果的实际意义}

该计算结果具有一定的实际意义,可作为分析胃溃疡病理的重要参考。然而,题目给出的样本量相对较小,得出的结论不一定准确,在实际情况下,应当增大样本量,并保证原始数据准确无误,从而得出更可靠的结论。

\subsubsection{结论}

使用病人组的全部数据计算得出,病人与正常人的溶菌酶含量有显著差别;去掉病人组的最后5个数据后计算得出,病人与正常人的溶菌酶含量无显著差别。
