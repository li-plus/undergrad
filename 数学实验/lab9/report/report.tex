% !TEX program = xelatex

\documentclass[12pt,a4paper]{article}
\usepackage[UTF8]{ctex}
\usepackage{float}
\usepackage{amsmath}
\usepackage{amsfonts}
\usepackage{enumerate}
\usepackage{booktabs}
\usepackage{graphicx}
\usepackage{longtable}
\usepackage{subfigure}

\usepackage{url}
\usepackage{multirow}

% for plotting 
\usepackage{caption}
\usepackage{pgfplots}

% for pseudo code 
\usepackage{algorithm}
\usepackage[noend]{algpseudocode}

% for reference 
\usepackage{hyperref}
\usepackage{cleveref}

% for code 
\usepackage{listings}
\usepackage{xcolor}
\usepackage{fontspec}
\definecolor{darkgreen}{rgb}{0,0.6,0}
\newfontfamily\consolas{Consolas}

\usepackage{amssymb}
\usepackage{pifont}
\usepackage{xcolor}
\newcommand{\cmark}{\ding{51}}
\newcommand{\xmark}{\ding{55}}

\lstset {
    basicstyle=\footnotesize\consolas, % basic font setting
    breaklines=true, 
    frame=single,     % {single, shadowbox, bottomline}
    keywordstyle=\color{blue}, 
    commentstyle=\color{darkgreen},
    stringstyle=\color{red},
    showstringspaces=false,
    % backgroundcolor=\color{black!5}, % set backgroundcolor
    numbers=left, 
    numberstyle=\ttfamily,
}

% Microsoft Word A4 paper default layout 
\usepackage[a4paper, left=3.18cm, right=3.18cm, top=2.54cm, bottom=2.54cm]{geometry}

% \captionsetup[figure]{labelfont={bf}, name={Figure}}
% \captionsetup[table]{labelfont={bf}, name={Table}}

\crefname{equation}{方程}{方程}
\Crefname{equation}{方程}{方程}
\crefname{table}{表}{表}
\Crefname{table}{表}{表}
\crefname{figure}{图}{图}
\Crefname{figure}{图}{图}

\title{数学实验:第九次作业}
\author{计算机系 \quad 计73 \quad 2017011620 \quad 李家昊}
\date{\today}

% 实验报告格式的基本要求

% 系别、班级、学号、姓名

% 1 实验目的
% 2 题目
%   2.1 计算题:题号,算法设计(包括计算公式),程序,计算结果(计算机输出),结果分析,结论。
%   2.2 应用题:题号,问题分析,模型假设,模型建立,算法设计(包括计算公式),程序,计算结果(计算机输出),结果的数学分析,结果的实际意义,结论。
% 3 收获与建议

% Calc
% \subsubsection{算法设计}
% \subsubsection{程序}
% \subsubsection{计算结果}
% \subsubsection{结果分析}
% \subsubsection{结论}

% App
% \subsubsection{问题分析}
% \subsubsection{模型假设}
% \subsubsection{模型建立}
% \subsubsection{算法设计}
% \subsubsection{程序}
% \subsubsection{计算结果}
% \subsubsection{结果的数学分析}
% \subsubsection{结果的实际意义}
% \subsubsection{结论}

\begin{document}

\maketitle

\section{实验目的}

\begin{itemize}
    \item 掌握数据的参数估计、假设检验的基本原理、算法,及用MATLAB实现的方法。
    \item 练习用这些方法解决实际问题。
\end{itemize}

\section{问题求解}

\subsection{Chap12-Ex5 货物抽检(应用题)}

% 5. 甲方向乙方成批供货,甲方承诺合格率为90\%,双方商定置信概率为95\%。现从一批货中抽取50 件,43 件为合格品,问乙方应否接受这批货物? 你能为乙方不接受它出谋划策吗?

\subsubsection{问题分析}

题目给出了产品承诺的合格率和置信概率,以及一次抽检的结果,需要确定该产品是否达标,这是一个0-1分布总体均值的假设检验问题。

\subsubsection{模型假设}

为了简化实际情况,模型基于以下假设,
\begin{enumerate}
    \item 每件产品是否合格服从0-1分布。
    \item 每批货物的数量充分大。
\end{enumerate}

\subsubsection{模型建立}

根据中心极限定理,当样本量充分大时,样本的均值近似服从正态分布。根据假设,每批货物的数量充分大,因此可对这批货物总体的合格率$p$做正态总体均值的假设检验,记甲方承诺的合格率为$p_0$,考虑到对合格率的检查应当采用单侧检验,则原假设$H_0$和对立假设$H_1$为,
\begin{equation}
    H_0: p \ge p_0, \quad H_1: p < p_0
\end{equation}

对上述假设进行$z$检验,记样本容量为$n$,合格率为$\overline{x}$,显著性水平为$\alpha$,$N(0,1)$的$\alpha$分位数为$u_\alpha$,并且,
\begin{equation}
    z = \frac{\overline{x} - p_0}{\sqrt{p_0(1-p_0)/n}}
\end{equation}

则当$z \ge u_\alpha$时,$H_0$被接受,乙方应当接受这批货物;否则,$H_0$被拒绝,乙方不应当接受这批货物。

从上述理论分析可以看出,作为乙方,如果不想接受这批货物,即拒绝$H_0$,有以下几种可能的策略。
\begin{enumerate}
    \item 与甲方商定新的置信概率$1-\alpha$。
    \item 要求甲方调节产品合格率$p_0$。
    \item 若样本合格率$\overline{x}$保持不变,可调节样本容量$n$。
\end{enumerate}

\subsubsection{算法设计}

对于$z$检验,可采用\texttt{ztest}命令。

\subsubsection{程序}

请参见附录\ref{sec:ex5_code}。

\subsubsection{计算结果}

在双方商定的95\%置信概率下,经过$z$检测,原假设被接受,得到$p$值为0.1729,置信区间为$(-\infty,0.9298]$,因此,乙方应当接受这批货物。

作为乙方,如果不想接受这批货物,可以采取以下其中一种策略。
\begin{enumerate}
    \item 与甲方商定新的置信概率,不得超过82.71\%。
    \item 要求甲方提高产品合格率,不得低于92.23\%。
    \item 若样本合格率保持不变,可以增大抽检样本容量,不得低于154。
\end{enumerate}

\subsubsection{结果的数学分析}

对于置信概率的调节,可以利用已经求得的$p$值,使置信概率低于$1-p$,此时显著性水平$\alpha$将高于$p$值,原假设将被拒绝。

对于产品合格率和抽检样本容量的调节,需要在其它变量固定不变时,求解方程$z < u_\alpha$。

\subsubsection{结果的实际意义}

该计算结果具有一定的实际意义,可作为乙方制定采购策略的重要参考。在实际应用中,只要保证随机抽样,且样本量足够大,就应当按照商定好的置信概率行事,这样对双方都是有利的。

如果抽检样本的合格率低于产品合格率,乙方应当接受但实在不愿意接受时,应当采取第3种策略来进一步验证,即加大抽检量,此时一般有两种情况:第一,这批货物的总体合格率是符合承诺标准的,那么随着抽检量的增加,抽检样本的合格率应该逐步接近或高于承诺的合格率,原假设始终被接受,乙方应当接受这批货物;第二,这批货物的总体合格率不符合标准,那么随着抽检量的增加,原假设必然在某一时刻被拒绝,乙方可以不接受这批货物。

\subsubsection{结论}

乙方应当接受这批货物。若乙方不想接受,可以与甲方商定,将置信概率降低到82.71\%以下,或者将合格率提高到92.23\%以上,或者将抽检样本容量提高到154以上。


\subsection{Chap12-Ex6 身高体重(计算题)}

% 6. 学校随机抽取100名学生,测量他们的身高和体重,所得数据如表12.6。
% 1)	对这些数据给出直观的图形描述,检验分布的正态性;
% 2)	根据这些数据对全校学生的平均身高和体重做出估计,并给出估计的误差范围;
% 3)	学校10年前作过普查,学生的平均身高为167.5厘米,平均体重为60.2公斤,根据这次抽查的数据,对学生的平均身高和体重有无明显变化做出结论。

\subsubsection{算法设计}

\paragraph{第(1)问} 首先检验总体分布的正态性,题目给出了100名学生的身高和体重数据,样本量不小,可采用Jarque-Bera检验和Lilliefors检验,对应的MATLAB命令分别为\texttt{jbtest}和\texttt{lillietest};然后进行数据可视化,使用\texttt{histfit}命令画出频率分布直方图并拟合正态曲线。

\paragraph{第(2)问} 对于正态分布的参数估计,可使用\texttt{normfit}命令,根据给定的显著性水平,对总体均值$\mu$和方差$\sigma$进行点估计和区间估计。

\paragraph{第(3)问} 题目给出了10年前的总体均值$\mu_0$,采用单总体均值的假设检验方法,记原假设$H_0$和对立假设$H_1$为,
\begin{equation}
    H_0: \mu = \mu_0, \quad H_1: \mu \ne \mu_0
\end{equation}

由于总体方差未知,故采用$t$检验,对应的MATLAB命令为\texttt{ttest}。若原假设$H_0$被接受,则10年来学生身高体重无明显变化,反之,则发生了显著变化。

\subsubsection{程序}

请参见附录\ref{sec:ex6_code}。

\subsubsection{计算结果}

\paragraph{第(1)问} 在默认的显著性水平(0.05)下,Jarque-Bera检验和Lilliefors检验均表明,身高和体重两个总体均服从正态分布。作出频率分布直方图及拟合的正态曲线,如\Cref{fig:ex6_histfit},可以看出,身高体重的频率分布基本符合正态分布曲线。

\begin{figure}[H]
    \centering
    \subfigure[身高分布]{
        \includegraphics[width=0.47\textwidth,trim={3.09cm 9.295cm 3.09cm 9.295cm},clip]{fig/ex6_height_histfit.pdf}
    }
    \subfigure[体重分布]{
        \includegraphics[width=0.47\textwidth,trim={3.09cm 9.295cm 3.09cm 9.295cm},clip]{fig/ex6_weight_histfit.pdf}
    }
    \caption{学生身高和体重的频率分布直方图及拟合的正态曲线}
    \label{fig:ex6_histfit}
\end{figure}

\paragraph{第(2)问} 在默认的显著性水平(0.05)下,身高和体重两个总体的参数估计如\Cref{tab:ex6_normfit},其中包括均值和方差的点估计和区间估计。在不同的显著性水平下,身高和体重的置信区间如\Cref{fig:ex6_height_alpha}和\Cref{fig:ex6_weight_alpha}。

\begin{table}[H]
    \centering
    \caption{身高和体重分布的参数估计}
    \label{tab:ex6_normfit}
    \begin{tabular}{c|ccccc}
        \toprule
        指标 & 均值点估计 & 标准差点估计 & 均值区间估计 &
        标准差区间估计\tabularnewline
        \midrule
        身高 (cm) & 170.25 & 5.40 & [169.18, 171.32] & [4.74,
        6.28]\tabularnewline
        体重 (kg) & 61.27 & 6.89 & [59.90, 62.64] & [6.05,
        8.01]\tabularnewline
        \bottomrule
    \end{tabular}
\end{table}

\begin{figure}[t]
    \centering
    \subfigure[均值置信区间]{
        \includegraphics[width=0.47\textwidth,trim={3.09cm 9.295cm 3.09cm 9.295cm},clip]{fig/ex6_height_mu_alpha.pdf}
    }
    \subfigure[方差置信区间]{
        \includegraphics[width=0.47\textwidth,trim={3.09cm 9.295cm 3.09cm 9.295cm},clip]{fig/ex6_height_sigma_alpha.pdf}
    }
    \caption{不同的显著性水平下,身高的总体均值和方差的置信区间}
    \label{fig:ex6_height_alpha}
\end{figure}

\begin{figure}[t]
    \centering
    \subfigure[均值置信区间]{
        \includegraphics[width=0.47\textwidth,trim={3.09cm 9.295cm 3.09cm 9.295cm},clip]{fig/ex6_weight_mu_alpha.pdf}
    }
    \subfigure[方差置信区间]{
        \includegraphics[width=0.47\textwidth,trim={3.09cm 9.295cm 3.09cm 9.295cm},clip]{fig/ex6_weight_sigma_alpha.pdf}
    }
    \caption{不同的显著性水平下,体重的总体均值和方差的置信区间}
    \label{fig:ex6_weight_alpha}
\end{figure}

\paragraph{第(3)问} 在默认的显著性水平(0.05)下,经过$t$检验,身高总体的原假设不成立,体重总体的原假设成立,由此可以断定,近10年来,学生的平均身高发生了明显变化,而平均体重无明显变化。

\subsubsection{结果分析}

从\Cref{fig:ex6_histfit}可以看出,在较大的样本量下,学生的身高和体重近似服从正态分布,这与概率论的中心极限定理相吻合。

从\Cref{fig:ex6_height_alpha}和\Cref{fig:ex6_weight_alpha}可以看出,当显著性水平趋于0时,置信区间长度趋于无穷;当显著性水平趋于1时,置信区间长度趋于0,收敛到点估计值;随着显著性水平的增加,即置信概率的降低,参数的置信区间随之缩小。计算结果与理论分析相符。

\subsubsection{结论}

在95\%的置信概率下,全校学生的身高和体重均服从正态分布,如\Cref{fig:ex6_histfit}。

全校学生的平均身高为170.25厘米,平均体重为61.27千克,在95\%的置信概率下,平均身高处于置信区间[169.18, 171.32]内,平均体重处于置信区间[59.90, 62.64]内。

在95\%的置信概率下,近10年来,学生的平均身高发生了明显变化,而平均体重无明显变化。


\subsection{Chap12-Ex7 胃溃疡病理(应用题)}

\subsubsection{问题分析}

题目设置了一个对照实验,分为胃溃疡病人和正常人两组,每组给出了30人的溶菌酶含量数据,需要确定病人和正常人的溶菌酶含量有无显著差别,这是一个两总体均值的假设检验问题。

\subsubsection{模型假设}

为了简化实际情况,模型基于以下假设,
\begin{enumerate}
    \item 正常人和胃溃疡病人的溶菌酶含量均服从正态分布。
    \item 除特别说明外,原始数据均准确无误。
\end{enumerate}

\subsubsection{模型建立}

由于样本量较小,未能通过总体分布的正态性检验,但是根据生活经验,很多医学现象都是服从正态分布的,因此这里假设病人和正常人的溶菌酶含量两个总体均服从正态分布,病人的溶菌酶含量服从$N(\mu_1, \sigma_1^2)$,正常人的溶菌酶含量服从$N(\mu_2, \sigma_2^2)$,使用两总体均值的假设检验方法,记原假设$H_0$和对立假设$H_1$为,
\begin{equation}
    H_0: \mu_1 = \mu_2, \quad H_1: \mu_1 \ne \mu_2
\end{equation}

由于总体方差未知,因此采用$t$检验。若原假设$H_0$被接受,则病人与正常人的溶菌酶含量无显著差别;若被拒绝,则有显著差别。

\subsubsection{算法设计}

在MATLAB的实现中,可采用\texttt{ttest2}命令进行两总体均值的假设检验。

\subsubsection{程序}

请参见附录\ref{sec:ex7_code}。

\subsubsection{计算结果}

使用病人组的全部数据时,用点估计求得总体分布为$N(15.33,16.45^2)$。经过$t$检验,得到$p$值为0.0251,低于默认的显著性水平0.05,原假设被拒绝,表明病人与正常人的溶菌酶含量有显著差别。

去掉病人组的最后5个数据后,用点估计求得总体分布为$N(11.51,11.75^2)$。经过$t$检验,得到$p$值为0.1558,高于默认的显著性水平0.05,原假设被接受,表明病人与正常人的溶菌酶含量无显著差别。

\subsubsection{结果的数学分析}

经过计算,病人组的最后5个数据很可能有误,因为其中包含了2个$3\sigma$之外,1个$2\sigma$之外的离群点,使参数估计产生了较大误差。为了减小误差,应当尽可能降低原始数据的噪声。

\subsubsection{结果的实际意义}

该计算结果具有一定的实际意义,可作为分析胃溃疡病理的重要参考。然而,题目给出的样本量相对较小,得出的结论不一定准确,在实际情况下,应当增大样本量,并保证原始数据准确无误,从而得出更可靠的结论。

\subsubsection{结论}

使用病人组的全部数据计算得出,病人与正常人的溶菌酶含量有显著差别;去掉病人组的最后5个数据后计算得出,病人与正常人的溶菌酶含量无显著差别。


\section{收获与建议}

在本次实验中,我掌握了数据的参数估计方法,以及假设检验的基本原理和算法,用统计推断方法建立了实际问题的模型,并用MATLAB进行求解,在解决实际问题的过程中,我对数学方法的原理和应用有了更深刻的理解。

希望助教能对每次的实验进行详细的解答,希望老师在未来的课堂上介绍更多数学应用的前沿知识。

\section{附录:程序代码}

\subsection{Chap12-Ex5}\label{sec:ex5_code}

\lstinputlisting[language=Matlab]{../src/ex5.m}

\subsection{Chap12-Ex6}\label{sec:ex6_code}

\lstinputlisting[language=Matlab]{../src/ex6.m}

\subsection{Chap12-Ex7}\label{sec:ex7_code}

\lstinputlisting[language=Matlab]{../src/ex7.m}

\end{document}