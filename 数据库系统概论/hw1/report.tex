% !TEX program = xelatex
% !BIB program = bibtex

\documentclass[12pt,a4paper]{article}
\usepackage[UTF8]{ctex}
\usepackage{float}
\usepackage{amsmath}
\usepackage{amsfonts}
\usepackage{enumerate}
\usepackage{booktabs}
\usepackage{graphicx}
\usepackage{longtable}
\usepackage{subfigure}
\usepackage{multirow}
\usepackage{url}
\usepackage{hyperref}
\usepackage{cleveref}

% Microsoft Word A4 paper default layout 
\usepackage[a4paper, left=3.18cm, right=3.18cm, top=2.54cm, bottom=2.54cm]{geometry}

\title{数据库系统概论:第一次作业}
\author{2017011620 计73 李家昊}
\date{\today}

\begin{document}

\maketitle

\section{论文阅读}

阅读论文 A Relational Model of Data for Large Shared Data Banks \cite{codd2002relational},其中提出了关系型数据库模型。

\section{问题回答}

\paragraph{Q1: What is the notion of data independence? Why is it important?}~{}

数据独立性是指:应用程序和终端活动独立于数据类型的增长和数据表示的变化。数据独立性有利于将上层应用程序和底层数据结构解耦,构造通用型数据库,简化应用程序的编写。

\paragraph{Q2: Codd spends a fair amount of time talking about "Normal forms." Why is it important that a database be stored in a normal form?}~{}

数据库存储为 Normal Form,即将复杂结构转换为简单结构,使得每个域都是简单域,有利于:

\begin{enumerate}
    \item 简化数据结构,可用数组存储,无需引入指针、哈希、索引和顺序列表。
    \item 简化数据项的名字,只需指定关系名和域名就能确定一类数据。
    \item 开发一种通用的数据描述语言,并且一阶谓词逻辑就能满足需求。
\end{enumerate}

\bibliographystyle{plain}
\bibliography{report}

\end{document}